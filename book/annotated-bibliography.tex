\def\book#1{\bigskip\textbf{#1}\par}%
\chapter{Annotated Bibliography}

\book{Free Market Fairness by John Tomasi}

In some ways, this is the same book as this one (but better written). It is a
philosophical treatment of the themes of this book. Its main argument is that
market friendly systems (market democracy) can satisfy Rawls' difference
principle while protecting a \emph{larger set of basic rights} than the social
democratic systems that Rawls himself preferred.

\book{Is the Welfare State Justified? by Daniel Shapiro}

Another philosophical treatment of the same questions, but with a stronger
emphasis on the argument that free market solutions are empirically more likely
to work, as opposed to the book above, which stays on the ideal-theory level.

\book{The Darwin Economy by Robert Frank}

The working title for \emph{The Darwin Economy} was \emph{The Libertarian
Welfare State}, but thankfully for me, Robert Frank changed his mind (also his
new title is better for his book and Libertarian Welfare State is better for my
book).

This is a more pop-economics treatment of its subjects. It is also a book that
has a more left-wing feel than my own. It often suffers from the \emph{curious
asymmetry} and, in many other areas, I feel that its conclusions do not follow
from the arguments. Still, in the end, there is a lot of agreement on policies:
that taxes are often better than regulation (for both liberals and
libertarians), that a progressive consumption tax has a lot of advantages, and
others. We should always be on the lookout for the possibility of policy
agreement even if we disagree on the fundamental justification of those
policies.


\book{Declaration of Independents by Nick Gillespie and Matt Welch}

This is a very different book. Doing justice to its title, it reads more like
an old-school pamphlet from the Revolutionary Era than modern-day seemingly
detached punditry.

It is also devoid of much economics and focused on cultural issues, except a
few interesting historical tidbits, such as this quote from a debate between
Jimmy Carter and Ronald Reagan:

\begin{quote}
I share the basic belief of my region [against] an excessive government
intrusion into the private affairs of American citizens and also into the
private affairs of the free enterprise system. One of the commitments that I
made was to deregulate the major industries of this country.
\end{quote}

This was the closing statement of the Democratic candidate.

