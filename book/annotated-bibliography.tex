\chapter{Annotated Bibliography}\label{chpt:abiblio}
\def\book#1 by #2.{\bigskip\par\indent\indent\textbf{#1} by \textit{#2}\par}%
\medskip

\book Libertarianism: What Everyone Needs to Know by Jason Brennan.

``What everyone needs to know'' is a series by Oxford University Press, of
which this is the libertarianism contribution.

The book is organized around a series of questions and answers. Most of the
discussion involves dispelling myths about libertarians. I have discussed the
Curious Asymmetry where anyone who argues for less government finds themselves
accused of wishing to abolish the state, when what most of us want is a little
less intervention. A large fraction of this book tackles these issues head-on.

Jason Brennan is a bleeding-heart libertarian (a libertarian who advocates a
freer market as the best solution for the worse-off), but he tries to give
adequate space to \emph{hard libertarianism} (the more anarchists strand of
libertarianism) and to classical liberals (moderate libertarians). He also
correctly emphasizes public choice considerations and not an unbinding faith in
the free market as one of the most important (maybe the most important) driver
of skepticism about state-driven mechanisms. As the GMU economist Gareth Jones
summarized, his thinking is neither ``markets fail, use the government;'' nor
``government fails, use the market;'' rather ``markets fail, government fails;
use markets'' (because their failures are less damaging and more likely to
self-correct).

Here are a few choice quotes from \emph{Libertarianism: What Everyone Needs to
Know}: ``most libertarians throughout history have advocated libertarianism in
part out of concern for the poor;'' ``libertarians and conservatives both talk
free-market talk, but libertarians mean it;'' and, one which could have come
from writings: ``many other European countries might reasonably be considered
more economically libertarian than the United States.'' The countries mentioned
in the context of the last quote are the usual suspects: Holland, Denmark,
Luxembourg, the UK, and Switzerland; which combine free-markets with a
market-driven (or, as I would write, libertarian) welfare state. I would
perhaps omit the UK (as too left-wing) and add Sweden.

\book Free Market Fairness by John Tomasi.

In some ways, Free Market Fairness the same book as the one you are reading,
but at a different level. This is a book of political philosophy.

Its main argument is that market friendly systems (what the author calls
\emph{market democracy}) can satisfy Rawls' difference principle while
protecting a \emph{larger set of basic rights} than the social democratic
systems that Rawls himself preferred.

\FIXME{This was written when I was being very succint. Needs more}

\book Is the Welfare State Justified? by Daniel Shapiro.

This is another philosophical treatment of why certain philosophical arguments
for the classical, command and control welfare state, can be used to justify
more libertarian solution. Compared with Free Market Fairness, the book by D.\
Shapiro has a stronger emphasis on the argument that free market solutions are
empirically more likely to work and will often resort to empirical research to
argue its points.

\book The Darwin Economy by Robert Frank.

The working title for \emph{The Darwin Economy} was \emph{The Libertarian
Welfare State}, but thankfully for me, Robert Frank changed his mind (also his
new title is better for his book and Libertarian Welfare State is better for my
book).

This is a more pop-economics treatment of its subjects. It is also a book that
has a more left-wing feel than my own. It often suffers from the \emph{curious
asymmetry}\footnote{This is the notion that compared centre-left liberals to
totalitarian communists is idiotic, but arguments from the free-market side are
often answered by responding to extreme views (e.g., if there was no
government, others would immediately invade, therefore my favorite policy is
the right thing). The search for middle ground will be a recurrent theme in the
book.} and, in many other areas, I feel that its conclusions do not follow from
the arguments. Still, in the end, there is a lot of agreement on policies: that
taxes are often better than regulation (for both liberals and libertarians),
that a progressive consumption tax has a lot of advantages, and others. We
should always be on the lookout for the possibility of policy agreement even if
we disagree on the fundamental justification of those policies. Here is R.\ Frank
on CO$_2$ taxes: ``As John Stuart Mill maintained, the government may
legitimately restrict individual behavior to prevent undue harm to others. But
heavy-handed regulation is almost never the most effective tool to that end.''

\book Declaration of Independents by Nick Gillespie and Matt Welch.

This is a very different book. Doing justice to its title, it reads more like
an old-school pamphlet from the Revolutionary Era than modern-day seemingly
detached punditry.

It is also devoid of much economics and focused on cultural issues, except a
few interesting historical tidbits, such as this quote from a debate between
Jimmy Carter and Ronald Reagan:

\begin{quote}
I share the basic belief of my region [against] an excessive government
intrusion into the private affairs of American citizens and also into the
private affairs of the free enterprise system. One of the commitments that I
made was to deregulate the major industries of this country.
\end{quote}

This was the closing statement of the Democratic candidate.

\book A Capitalism for the People by Luigi Zingales.

Luigi Zingales focus not on government models \emph{per se}, but on the
relationship of government to private business. He argues that the framing of
discussion as one of government versus big business is a left-over from 20th
century ideological battles. They may now hurt rather than help both the
intelectual discussion and the policy results.

This book is necessary as the free-market advocates have too often let
themselves be bought and defend business interests (especially big-business)
instead of free markets. In fact, they have often become opposites as most
corporate lobbying is for more regulation (naturally, regulation of a certain
kind) and subsidies.

I do not agree with all this book says, but the solutions that L.\ Zingales
presents are remarkably similar to those in this book.

\book Catastrophic Care: How American Health Care Killed My Father and How We
Can Fix It by David Goldhill.

This is a wonderful book, which expands on an essay of the same name. Unlike
many books that started as essays, this one adds detail, depth, and breadth to
the original essay. David Goldhill, who describes himself as Democrat, clearly
shares the values and goals of liberals who wished to expand health insurance
coverage to all Americans. However, he is skeptical of whether the tools that
the political system has found (in particular, the Affordable Care Act, aka
Obamacare) will have the desired results.

I found much to agree and little to disagree with this book. I particularly
enjoyed the description of ``Island thinking'': how healthcare discussion
happens in isolation of how the rest of society is connected to the healthcare
island.

Here are a few quotes from the book: ``all of the [current] complexity [of the
health care system] exists for only one reason: to maintain the fiction that
someone else is paying for your health care.'' ``My answer to those who argue
for a national health insurance or for a market-based system is to do both'' (a
truly ``libertarian welfare state" thought---from a Democrat, which just shows
how much our political dichotomies can be false dichotomies).

