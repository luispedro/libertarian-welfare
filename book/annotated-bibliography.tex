\medskip
\medskip
\def\book#1 by #2.{\bigskip\par\indent\indent\textbf{{\color{pondgreen}#1} by \textit{\color{pondgreen}{#2}}}\par}%
\noindent{\sechead\color{toastedchilipowder}{Annotated Bibliography}}
\medskip

\book Free Market Fairness by John Tomasi.

In some ways, this is the same book as this one, but at a different level. It is a
philosophical treatment of the themes of this book. Its main argument is that
market friendly systems (market democracy) can satisfy Rawls' difference
principle while protecting a \emph{larger set of basic rights} than the social
democratic systems that Rawls himself preferred.

\book Is the Welfare State Justified? by Daniel Shapiro.

Another piece of the elephant, this is philosophical treatment of why certain
philosophical arguments for the classical, command and control welfare state,
can be used to justify more libertarian solution. Compared with Free Market
Fairness, the book by D.\ Shapiro has a stronger emphasis on the argument that
free market solutions are empirically more likely to work and will often resort
to empirical research to argue its points.

\book The Darwin Economy by Robert Frank.

The working title for \emph{The Darwin Economy} was \emph{The Libertarian
Welfare State}, but thankfully for me, Robert Frank changed his mind (also his
new title is better for his book and Libertarian Welfare State is better for my
book).

This is a more pop-economics treatment of its subjects. It is also a book that
has a more left-wing feel than my own. It often suffers from the \emph{curious
asymmetry}\footnote{This is the notion that compared centre-left liberals to
totalitarian communists is idiotic, but arguments from the free-market side are
often answered by responding to extreme views (e.g., if there was no
government, others would immediately invade, therefore my favorite policy is
the right thing). The search for middle ground will be a recurrent theme in the
book.} and, in many other areas, I feel that its conclusions do not follow from
the arguments. Still, in the end, there is a lot of agreement on policies: that
taxes are often better than regulation (for both liberals and libertarians),
that a progressive consumption tax has a lot of advantages, and others. We
should always be on the lookout for the possibility of policy agreement even if
we disagree on the fundamental justification of those policies. Here is R.\ Frank
on CO$_2$ taxes: ``As John Stuart Mill maintained, the government may
legitimately restrict individual behavior to prevent undue harm to others. But
heavy-handed regulation is almost never the most effective tool to that end.''

\book Declaration of Independents by Nick Gillespie and Matt Welch.

This is a very different book. Doing justice to its title, it reads more like
an old-school pamphlet from the Revolutionary Era than modern-day seemingly
detached punditry.

It is also devoid of much economics and focused on cultural issues, except a
few interesting historical tidbits, such as this quote from a debate between
Jimmy Carter and Ronald Reagan:

\begin{quote}
I share the basic belief of my region [against] an excessive government
intrusion into the private affairs of American citizens and also into the
private affairs of the free enterprise system. One of the commitments that I
made was to deregulate the major industries of this country.
\end{quote}

This was the closing statement of the Democratic candidate.

