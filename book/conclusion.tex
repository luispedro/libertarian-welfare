\chapter{Conclusions}

The cause of liberty and free-markets has too often been abused and coopted by
those who do not truly care about it, but instead just want to maintain
privilege. This has won it support from some quarters and maybe even some
political victories, but it comes at the cost of conceding the moral point to
the opponents of free markets. In recent years, ``bleeding heart libertarians''
have become more vocal, but they can trace their intelectual heritage to The
giants of libertarianism such as Milton Friedman or Friedrich Hayek (and
through them to Adam Smith).

I repeatedly argued that \emph{more economic freedom is better for the least
well-off}, that \emph{misregulation not deregulation is the problem}.

\section{What Was Left Out}

Some topics were left out of the book.

We restricted the discussion to rich countries. The poorer countries are almost
always much further from a libertarian ideal of respect for individuals and
property than richer ones (and because governments are weak and ineffective at
anything except extraction, it is a \emph{worst of both worlds} situation).

We did not discuss freer immigration, which is perhaps the best way to help the
global poor.\footnote{A really good initial starting point for this discussion
is the website \url{http://openborders.info/}, which is dedicated to making
this case.} This is a major omission as libertarianism is often and
historically \emph{cosmopolitan}. Recall that economics is called \emph{the
dismal science} because the first free-market economists committed what at the
time was considedered a moral mistake: they wanted to consider the interests of
black people and advocated ending slavery.\backnote{See this critique by Matt
Zwolinski of US-centric libertarianism:
\href{http://bleedingheartlibertarians.com/2013/07/libertarian-populism-and-libertarian-cosmopolitanism/}{Libertarian
Populism and Libertarian Cosmopolitanism}.}

\subsection{Some Libertarian Welfare Parting Thoughts}

\thought Libertarian welfare is about empowering the poor by giving them
resources directly.

\thought The strongest welfare state is a libertarian welfare state. This is
not a contradiction.

\thought The welfare state sometimes needs to be protected against its most
vocal defenders.

