\chapter{Conclusions}

The welfare state is one of the 20th century's achievements, but it suffers
from the vices of 20th century thinking. It is too trusting of bureacracies and
it mistrusts the very individuals it is trying to help.

The cause of liberty and free-markets has too often been abused and coopted by
those who do not truly care about it, but instead just want to maintain
privilege. This has won it support from some quarters and maybe even some
political victories, but it comes at the cost of what is best for lower-income
individuals. In recent years, ``bleeding heart libertarians'' have become more
vocal, but they can trace their intelectual heritage to The giants of
libertarianism such as Milton Friedman or Friedrich Hayek.

Some topics were left out of the book. We did not discuss freer immigration,
which is perhaps the best way to help the global poor.

\subsection{Some Libertarian Welfare Parting Thoughts}

\thought Libertarian welfare is about empowering the poor by giving them
resources directly.

\thought The strongest welfare state is a libertarian welfare state. This is
not a contradiction.

