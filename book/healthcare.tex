\chapter{Health Care}
\label{chpt:health}

Why is health care so expensive in the United States?

Most other industries are dominated by \emph{almost as good and so much
cheaper} competition. Health care is dominated by \emph{just slightly better
and so much more expensive} competition.

\section{Should Vioxx be Back on the Market?}

The received wisdom on Vioxx goes somewhat like this: Merck was making a lot of
money selling Vioxx. It turned out that it caused heart attacks, but they did
not tell anyone. Finally, a lawsuit made them take this medication off the
market. Almost all of this is true, except that I think that Vioxx should be
put back on the market.

Yes, it does cause a slight increase in the risk of heart attacks (so much is
true). However, the risk is slight. In fact, it is about the same risk as
taking Ibuprofen. Except that, unlike Ibuprofen, it has a lower risk of gastric
complications.

Patients should have the opportunity to make these decisions by themselves.

What about Merck hiding the data? The story is a bit more complex than this. I
think that the scientists at Merck could have been a bit more forthcoming.
However, this is not a reason to punish the patients. The scientists involved
could be punished by the journal with a publication ban (and their reputations
certainly took a hit). If there is anything wrong with the FDA filings (but
recall that all of the information was submitted to the FDA), the FDA could
take regulatory action and fine the company.

\section{Can the government buy healthcare?}

A common argument for heavy government intervention into the healthcare market
is that a market in healthcare is impossible for behavioral reasons: people do
not behave according to rational self-interest when it comes to health and this
makes the resulting market be skewed.\footnote{This argument has a strong
pedigree, it goes back to an article from Nobel laureate Kenneth Arrow,
published in 1963. In a distylled form, it is now very common in the mainstream
press.}

The reason is that health care is different from other markets. While anyone
can easily compare the price of apples or gas and it is something that we buy
on a regular basis, major health purchases are made in a situation of distress.
If they are made for family members, then it is morally impossible for a family
member to perform a cost-benefit analysis. Therefore, as individuals we choose
badly when it comes to health care.

However, it does not immediately follow that the government can do a better
job. We can ask the same question about government deciders: can they make
rational decisions in the context of healthcare? After all, government decisors
are people too and their motivations are not rational self-disinterest. The
political/bureacratic process has its own behavioral blind-spots. They
naturally differ from the blind spots of the individual, but are they any less
harmful?

For example, in the United States, the FDA had conditionally approved Avastin
for breast cancer. What this means is that it was still experimental, but there
was enough hope that it would have a positive effect in patient's outcomes that
it was OK to let the general patient population have access to the medication
while more studies were being conducted to confirm. Unfortunately, it turned
out that \emph{Avastin is actively harmful for breast cancer
patients}~\cite{avastin-harfmul}. The problem is that it seems that Avastin has
no effects on the cancer, but it has strong negative side-effects.

This led to a strong outcry. Those outcrying were wrong, of course, but this
fact did nothing to diminish their emotional appeal. Cancer survivors who had
taken Avastin appeared at televised hearings in tears (of course, thanks to
\textbf{other} drugs, many women now, thankfully, survive breast cancer; but
this is unrelated to Avastin).\footnote{See this note in the HufftingtonPost:
`` 'It's saved my life,' said a tearful Sue Boyce, 54, of Chicago.'',
\url{http://www.huffingtonpost.com/2011/11/18/avastin-breast-cancer-fda-approval-revoked_n_1101468.html}.}
Families of current patients and the patients themselves spoke about being
scared this drug will be taken away. The political system quickly responded and
provided guarantees that Medicare would continue to cover (pay for) Avastin.
Avastin costs 100,000 dollars a treatment and it hurts patients, but Medicare
rushed in to provide it at taxpayer expense.

Naturally, this could just be an isolated incidence. However, here are a few
other completely unscientific treatments that the political system has required
be covered:

\begin{itemize}
\item dolphin treatment
\item homeopathy?
\end{itemize}

Tyler Cowen likes to point out that ``mandates don't stay modest.''

In a similar vein, can the political system sustain a system of long-term
innovation or will it fall into the temptation of killing the goose that lays
the golden eggs? The FDA already errs on the side of too much caution and not
enough innovation.

\subsection{Should it be easier to practice medicine?}

This is one of those issues that instinctively leads to false dichotomies. I am
not proposing that anyone should be allowed to perform open heart surgery or
that we should have no barrier to acess to medical education. Think, however,
of the premed students who almost made it to medical school, but not quite.
They were surely ``doctor material.''

\subsection{Does the FDA save or lose lives?}

Obviously, it does both. We want to make it do more of the saving and less of
the losing, however.

A lot of innovation follows a familiar pattern, the first versions of anything
do not work very well, but are much cheaper. It's \emph{90\% of the effect and
10\% the cost}. Then, as the technology improves, the quality of the new
techniques will often not only match, but overcome what was possible with the
older technology. In medicine, most innovation, however, is \emph{10\% better
at 10 times the cost}. These are not just the musings of extreme
anarcho-capitalists. Michael Mandel at the Progressive Policy Institute has
made exactly these points in a brief on healthcare
(\url{http://autismcrisis.blogspot.pt/2007/08/autism-and-aba-in-uk-controlled-trial.html}).

\subsection{Healthcare Politics Thoughts}

``The individual is not always rationally self-interested, but the political
system is not rationally self-disinterested either.''

``Mandates don't stay modest.''
