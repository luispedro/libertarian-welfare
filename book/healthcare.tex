\chapter{Health Care}
\label{chpt:health}

Healthcare is perhaps the major topic that comes up with respect to the welfare
state. Perhaps because it is one that affects us all.

Much has been written about healthcare in the United States, especially over
the last few years with the passage of the Affordable Care Act (aka Obamacare)
and its step-by-step implementation. Each little step is scrutinized, with
minor changes being trumpeted as heralding a whole new world and every misstep
proving that the whole project is a failure. The only thing that is clear is
that the topic will continue to be active ovver a long time.

However, strangely, there has been little thought given to how to expand the
amount of health care provided. It is all about shuffling resources around, not
about creating more health care.

As the wonks say, the discussion has all been \emph{demand side} (who should
pay? how? when?) and not \emph{supply side} (who should provide services? how?
how to get more services provided?). This chapter will discuss how removing
obstacles to the provision of health care can result in cheaper healthcare, by
sacrificing the frills and using technology so that medical quality can be
maintained.

\section{Why is health care so expensive in the United States?}

Health care is expensive in the United States and it is not clear that
Americans are getting their money's worth. It is not clear that they are not,
either. It is very hard to compare health care results. We can compare health
outcomes, but most health outcomes depend most strongly on lifestyle choices.
Americans do not, on average, make the healthiest choices while eating, and
tend to live overly sedentary lives; until recently, they would smoke more than
almost everybody else too.\footnote{Even as I'm writing this, in 2013, the
levels of smoking in the United States are still at \FIXME.} They also drive
more, especially as teenagers, which shows up in the mortality statistics.
While the effects of widespread availability of guns on crime-related deaths
may be debatable, its effect on suicides is not. None of these factors are
related to health care.

Most other industries are dominated by \emph{almost as good and so much
cheaper} competition. Health care is dominated by \emph{just slightly better
and so much more expensive} competition. We need to focus on the first kind of
competition. To the extent that the extra benefits are not medical, but rather
luxury (plus waiting rooms) or feel-good (over educated doctors taking care of
routine issues), we should be happy to see them go and be replaced by cheaper
medical services.

There are two types of consumers of healthcare: the healthy and the sick. By
healthy I mean both people who are just looking for wellness services or
screening as well as people who are momentarily sick, but will soon be cured.
Of course, we may move from one group to the other (unfortunately, it is easier
to move from healthy to sick, but the reverse also happens). The needs of
these two groups are very different and we have to be careful not to sacrifice
one to the second.

Low deductibles certainly benefit the healthy, but the value of the deductible
is not a big consideration for a chronically sick individual; while the overall
cost is (and high deductibles can arguably lower the overall cost).

\section{Is the US healthcare system ``free market''?}

No, and the idea that it is is one of the most unhelpful ideas in the whole
debate. When I say that the system is not free market, I do not mean that the
system is not a perfect free, idealized free market. I mean that the healthcare
sector is one of the most heavily regulated, subsidized, and heavy-handed
managed in the United States. Therefore, to argue that its many failings (and
there are many failings, for sure) somehow show that the free market cannot
provide good healthcare is borderline ridiculous.

In fact, the US government \textbf{spends more on healthcare than other
\emph{so called socialist}\footnote{As I argue throughout the book, but
particular in Chapter~\ref{chpt:international}, many countries in Northern
Europe are not socialistic, but at least as free-market as the United States.}
healthcare systems}. This is without even counting the largest component of
government spending on healthcare: the tax deduction
on employer provided healthcare. If you add this component, then \emph{the US
government spends more on healthcare than any other developed country}. This is
not the hallmark of a free market. % FIXME Add numbers. Tables

The problem is not that the government is not involved in health care, the
problem is that it is involved in number of very negative ways (certainly also
in positive ways, but rarely is it uninvolved).

In fact, I think that there should be more pressure on free market economists
to explain why the US healthcare system actually works pretty well given this
deep involvement of the state.\footnote{This is partially tongue-in-cheek, but
I think this way to pose the problem has a lot of truth in it. My answer is
that health care is a highly desirable good. Therefore, people are willing to
pay the higher costs for all the extra layers and all the inefficiencies. The
cost of the state involvement does not appear in worse outcomes for most
people, but in higher costs for everyone and reduced access for a minority.}

The major feature of the American health insurance system, namely its linking
of coverage to employment is another result of government intervention. This is
one of those cases where there is wide agreement between left-wing and
right-wing researchers that this should be changed, and there is wide agreement
between left-wing and right-wing politicians that this should not be touched.
Obama repeatedly stressed that the health care reform he supported would not
impact anyone who already had health insurance.\footnote{As many politician
promises, this one contained enough truthiness to not be an outright lie, but
it was not really kept. According to the law, your company was allowed to keep
offering you the same health insurance as before, you would be grandfathered in
and would not be affected by any of the new regulations. Most companies,
however, decided that it was not worth the hassle of keeping two sets of plans,
one for the grandfathered clients, pre-Obamacare; another one,
which was Obamacare-compliant (in fairness, it was not very clear in exactly
what conditions they were allowed to do that, as they would then not be able to
make any changes in the plan). So, in fact, almost nobody will be able to keep
their health insurance as it was.}

In many markets in the US, to open a new hospital, you need the permission of
the existing hospitals. This is as far from a free market as you can possibly
get: it is using the government to impede competition. It is no surprise that
new hospitals will not get approved very often.

Naturally, the entry into the medical profession is also regulated by the
government. In fact, Congress more or less decides how many doctors to train
each year. Which is to say, how many doctors will enter the profession that
year. They do this as they decide how many residency slots will be sponsored by
Medicare. As the residency experience is required for the practice of medicine,
this decides how many doctors will enter the profession. Although the
bottleneck is at this level, if it were larger, we would expect that medical
schools would soon start to train more doctors as well.

As usual, I do not argue for a complete deregulation of the doctors. It is
probably a good idea that there be certain checks on who can call themselves an
MD and practice medicine.\footnote{Although it is a good example of status quo
bias how we, at the same time, find it normal that quacks of all kinds be
allowed to perform sort-of-medicine as long as they do not call it medicine. A
homeopath can claim to heal any disease with his magic water, often making a
living off exploiting the sick, and we find it completely normal.} However,
there can be too much of a good thing. The smart premed who just misses his
entry into med school would probably become a fine doctor if we gave him a
chance.

\subsection{But it is the most free market system, right?}

You might be thinking that ``even if the US is not so free-market, it is more
free-market than other countries.'' No, not really. Many other developed
countries have systems which are arguably more free-market than the US.

Again, other countries have less government spending on healthcare, they will
have fewer impediments to doctors practicing.

It is true that the system is often potraied as free-market by both its
defenders and its detractors. During the debates on health care reform, one of
the oft quoted lines from some of the more ignorant opposers was \emph{Keep
your government hands off my Medicare}. This is absurd: Medicare is a
government run program\footnote{And, no, I do not believe that the protestors
were arguing for a libertarian welfare solution of deregulating Medicare. Not
even in an unconscious way.}. This just another example of how free-market
rethoric is coopted and used to mean and advocate for decidely non-free-market
things. It actually points to the fact that healthcare in the US is already
mostly government run.

\section{Can the government buy healthcare?}

A common argument for heavy government intervention into the healthcare market
is that a market in healthcare is impossible for behavioral reasons: people do
not behave according to rational self-interest when it comes to health and this
makes the resulting market be skewed.\footnote{This argument has a strong
pedigree, it goes back to an article from Nobel laureate Kenneth Arrow,
published in 1963. In a restyled form, it is now very common in the mainstream
press.}

The reason is that health care is different from other markets. While anyone
can easily compare the price of apples or gas and it is something that we buy
on a regular basis, major health purchases are made in a situation of distress.
If they are made for family members, then it is morally difficult for a family
member to perform a cost-benefit analysis. Therefore, as individuals we choose
badly when it comes to health care.

However, it does not immediately follow that the government can do a better
job. We can ask the same question about government deciders: can they make
rational decisions in the context of healthcare? After all, government decisors
are people too and their motivations are not rational self-disinterest. The
political/bureaucratic process has its own behavioral blind-spots. They
naturally differ from the blind spots of the individual, but are they any less
harmful?

For example, in 2008, the FDA conditionally approved Avastin for use in breast
cancer. What this means is that it was still experimental, but there was enough
hope that it would have a positive effect in patient's outcomes that it was OK
to let the general patient population have access to the medication while more
studies were being conducted to confirm. Unfortunately, it turned out that
\emph{Avastin is actively harmful for breast cancer
patients}~\cite{avastin-harfmul}\footnote{See this note in Science magazine:
\url{http://news.sciencemag.org/scienceinsider/2011/06/breast-cancer-drug-gets-a-unanimous.html}}.
The problem is that it seems that Avastin has no effects on the cancer, but it
has strong negative side-effects.

This led to a strong outcry. Those outcrying were wrong, of course, but this
fact did nothing to diminish their emotional appeal. Cancer survivors who had
taken Avastin appeared at televised hearings in tears (of course, thanks to
\textbf{other} drugs, many women now, thankfully, survive breast cancer; but
this is unrelated to Avastin).\footnote{See this note in the HufftingtonPost:
`` 'It's saved my life,' said a tearful Sue Boyce, 54, of Chicago.''\backnote{
\url{http://www.huffingtonpost.com/2011/11/18/avastin-breast-cancer-fda-approval-revoked_n_1101468.html}.}}
Families of current patients and the patients themselves spoke about being
scared this drug will be taken away. The political system quickly responded and
provided guarantees that Medicare would continue to cover (pay for) Avastin.
Avastin costs 100,000 dollars a treatment and it hurts patients, but Medicare
rushed in to provide it at taxpayer expense.

This is not just one isolated incident. As we discuss in more detail below, the
problem is more profound: every time a government regulator stops a good
solution, the victims are invisible or seem unconnected to the regulation; so
there is little penalty in making this sort of mistake. On the other hand, if
it ever allows anything that is later revealed to have been a mistake or even
possibly a mistake, then there is an outcry. Therefore, \emph{government
regulators should be expected to be very risk adverse}.

A separate issue with government regulation (particularly in the American
political system), is that it responds strongly to well-organized pressure.
This leads it to cave to interest groups that manage to garner both money and
sympathy. This is particularly true if it is regulating what others must spend.
Here are a few other completely unscientific treatments that the political
system has required be covered by private health insurance:

\begin{itemize}
\item dolphin treatment
\item acupuncture\backnote{\emph{Acupuncture as ‘essential’ health care?
California weighs the question} by Sarah Kliff, Washington Post Wonkbook blog,
September 1, 2012.
\href{link}{http://www.washingtonpost.com/blogs/wonkblog/wp/2012/09/01/acupuncture-as-essential-health-care-california-weighs-the-question/}.}
\item homeopathy?
\end{itemize}

Tyler Cowen likes to point out that ``mandates don't stay
modest.''\footnote{This implies an argument for public provision which I find
rather strong: the government as a provider has a stronger interest not to
provide overly expensive treatments; while as a regulator, it can mandate them
as the political/media system suffers from a blind-spot in this area as it
fails to immediately point out that the cost will come out of rising premia.}
Furthermore, as David Goldhill remarks, \emph{health insurance companies
benefit from mandates}. While they may wish to keep their current payments low
(and thus fight having to cover a specific treatment), over the long term, the
more coverage they are forced to provide the better for them as this will be
passed on as premia (because all insurance companies are bound by these
mandates, competition will not keep prices low).

This just shows how the typical narrative of a courageous legislature imposing
a mandate on an unwilling insurance industry has it backwards. To misquote a
diplomat, lobbying is the art of having the legislation you want imposed on
you.\footnote{The original quote reads ``diplomacy is the art of letting
someone have your way'' and is attributed to Daniele Vare, an Italian diplomat
and novelist.}

The reverse is true for individual incentives. We all want to get the most of
the insurance we already paid for. We want to argue and fight for that little
extra benefit. Note how even the language tricks us: everything that an
insurance pays for is called a \emph{benefit}, even if it does very little for
our health. Over the long-term, though, whatever is spent, even if of dubious
medical value or sold at inflated prices, will come out of our pockets.

Can the political system sustain a system of long-term innovation or will it
fall into the temptation of killing the goose that lays the golden eggs? The
FDA already errs on the side of too much caution and not enough innovation.

As with many other governemnt provided services, the government bureacracies
will tend to treat certain kinds of beneficiaries better than others.
Recently, the first Obamacare forms appeared. They are 21~pages and resemble a
tax form. As Megan McArdle pointed out, ``since the whole point is to award
subsidies on the basis of your income, that's not exactly
surprising.''\backnote{\emph{Applying for Obamacare Subsidies Will Be as
Complicated as Doing Your Taxes} by Megan McArdle in The Daily Beast
\url{http://www.thedailybeast.com/articles/2013/03/14/applying-for-obamacare-subsidies-will-be-as-complicated-as-doing-your-taxes.html}.}
While some of the complexity may be inherent to the system, it does itself form
a sort of filter. It privileges those who are better able to deal with complex
forms.\footnote{Non-money elites (journalists, academics, or any other
high-status, low-pay professionals) are often suspect when they promote more
government involvement. They are the ones who will most able to capture the
bulk of government benefits. They will be able to get information on which
benefits are worth it, and apply correctly. They will also be better treated by
public workers than other beneficiaries.\FIXME{Look up study that upper
middle-class gets better drugs from doctors in British NHS.}}

To summarize:
\begin{itemize}
\item The government is too responsive to emotional appeals by well-organized
groups of patients, independently of whether science is on their side.
\item The government takes into account the interests of existing suppliers.
\item Government services often exhibit an inequality of access that is
non-monetary, but no less real and damaging.
\end{itemize}

Maybe individuals are not the best customers of health care, or of anything for
that matter. They may still be the best we've got. Remember the DC dictum:
\emph{those who do not sit at the table, find themselves on the menu}. And
individuals rarely get to sit at the table.

\subsection{Should it be easier to practice medicine?}

This is one of those issues that instinctively leads to false dichotomies. I am
not proposing that anyone should be allowed to perform open heart surgery
without any control or that we should have no barriers to the provision of
medical services. However, we can have too much of a good thing. Think,
however, of the premed students who almost made it to medical school, but not
quite.  They were surely ``doctor material.''

One of the interesting (if that is the right word) aspects of the medical
provision regulation is that it is, in fact, perfectly legal to sell snake oil
shams and many quaks get rich offering fake cures. They call themselves
homeopaths or naturopaths and happily take the money of sick people while
steering them away from the medical services that could help them.

\subsection{Does the FDA save or lose lives?}

Obviously, it does both. We want to make it do more of the saving and less of
the losing, however.

% Tiago Queiroz points out that a lot of innovation follows another pattern:
% initially very expensive, then costs come down
A lot of innovation follows a familiar pattern, the first versions of anything
do not work very well, but are much cheaper. It's \emph{90\% of the effect and
10\% the cost}. Then, as the technology improves, the quality of the new
techniques will often not only match, but overcome what was possible with the
older technology. In medicine, most innovation, however, is \emph{10\% better
at 10 times the cost}. These are not just the musings of extreme
anarcho-capitalists. Michael Mandel at the Progressive Policy Institute has
made exactly these points in a brief on healthcare
(\url{http://autismcrisis.blogspot.pt/2007/08/autism-and-aba-in-uk-controlled-trial.html}).

Still today, Vioxx is often brought up as a scandal and a failure of
regulation. However, Vioxx was not that awful: all pain drugs have terrible
side-effects, Vioxx was oversold as not having any, which was not true, but it
is no worse than any of the alternatives.\footnote{Still today, there are
arguments that \emph{Vioxx should be put back on the market} as some patients
might be better off managing the risks associated with it rather than go
without it. This is tangencial to my point, but I want to point out that the
idea can be defended.} Furthermore, even if we accept that Vioxx should not
have been sold, we need to be careful not to over-react to any mistake that is
only found a posteriori.

\section{Importing Healthcare}

One way in which the United States could lower its health care costs is by
importing more of it.

Recently, a robot-doctor gained approval in the United States. This is still
being touted as being centered on remote presence, which is a great way to get
doctors to hard-to-access regions.

% http://www.businesswire.com/news/home/20130124005134/en/FDA-Clears-Autonomous-Telemedicine-Robot-Hospitals

\subsection{Importing Healthcare Workers}

A study from the University of Virginia estimated that ``several US states
could solve physician shortages entirely by eliminating the additional
licensing requirements imposed on foreign educated
physicians.''\footnote{``Doctors With Borders: Occupational Licensing as an
Implicit Barrier to High Skill Migration'' by Brenton Peterson, Sonal S.
Pandyam, and David Leblang.}

Note that this would also open an avenue for Americans who could study abroad
for medical school. Already some do, but if it was easier to do so and return
home to practice, many more would perform this end-run around the high cost of
an American medical school.

\subsection{Consulate Care}

I end this discussion on \emph{importing healthcare} with a semi-whimsical
suggestion by George Mason polymath Robin Hanson: consulate care. To quote from
his blog \emph{Overcoming Bias}:

\begin{quote}
Let countries like Sweden, France, etc. with approved national health care
systems have bigger consulates, and open them up to paying customers for
medical services. For example, you could sign up for Swedish Care, and when
needed you’d go to their consulate to get medical care as if you were living in
Sweden.
\end{quote}

Although probably not something feasible (on both political and practical
grounds), this is an interesting thought experiment.

\section{Automating Healthcare}

We have discussed how certain tasks could be performed by someone with less
than a medical degree and how this process is already going on the in US with
nurse-practicioners. The ultimate form of down-crenditialing is automation.

When we replace an interaction with an expensive professional with an
interaction with a computer system, the costs for the user can be much lower.
Again, this has already taken place to some extent as websites such as WebMD or
Medline Plus from the National Institutes of Health have provided educated
persons with the possibility to check many of their symptoms and sometimes
replace a call or a doctor's visit.

In a recent edition of The Atlantic, Jonathan Cohn claims that ``nobody expects
American patients accustomed to treatment from live human beings to tolerate
such a sudden shift for much of their care.''\backnote{\emph{The Robot Will See
You Now} by Jonathan Cohn in The Atlantic, March 2013.} Of course not. Unless,
of course, you give them the choice and expose them to the costs thereof. The
robot being much cheaper, the cost for the patient will be much cheaper. In
fact, as long as patients are insulated from spending choices, we will expect
that they will not tolerate anything but what they perceive to be the best.

An auxiliary argument is that doctors play a role that is not just
physiological, but psychological. They do not only diagnose, ask for tests, and
prescribe remedies (pharmaceutical or otherwise); but they also play a human
role: they comfort and reassure.\footnote{Often, this is tied to a notion that
such comfort is also a form of healing, through the placebo effect. While the
placebo effect is, in fact, real, its effects are often exagerated as well. It
is often confused with the no-treatment effect: that is, for all but very
serious conditions, people tend to recover on their own. A second effect is
that the placebo effect is measurably greater for self-reported measures of
feeling than for physiologically measurable variables.} All of this is
undoubtedly true, but then the question is whether you really need someone with
over ten years of training, and in short demand, to serve that role. As it
stands, many people are already choosing to have some interactions with less
well-trained practicioners who we can afford to pay for longer session.  For
example, the resource to midwives or doulas by pregnant women is very often
motivated by the fact that it would be very expensive for them to pay an MD to
sit with them through the whole labor process, while they can afford a
less-well trained professional.

\section{What About Single-Payer?}

The British National Health Service is a single-payer system, meaning that the
government pays for and directly runs hospitals, clinics and most other
healthcare institutions. As a system, it is certainly based on good ideals, but
it is not an ideal system. The stories of failing to treat patients abound. A
major scandal made public in 2009 featured the sort of mistreatment of patients
that, had the system been privately run, would have been used to justify public
takeover.

It is also, contrary to claims, not a completely egalitarian system. In fact,
when alloating resources, doctors will provide higher income patients with
better treatments, even though they should only make needs-based
assessments\backnote{Find the studies for this claim.}. In this, it is not
unlike public education in most of the United States: everyone has formally
access to the same services, but there are tremendous differences in quality
which mean that those that are already better-off get better service.

\section{Catastrophic Insurance}

If there is one thing that scares people about health care it is that something
catastrophic could happen and destroy both their health and their finances.
They are afraid of a cancer diagnostic which will bring bills that they cannot
afford. In fact, unless you are particularly wealthy, the costs associated with
this type of disease will be overwhelmning.

The solution to this problem is to provide catastrophic insurance, ideally
linked to your recent income. What is a catastrophic expense to someone who
earns 20 thousand a year will not be the same as what is catastrophic to those
who earn that amount in a single month. This has the additional advantage of
keeping an active market even for fairly expensive therapies: at least some
people will still be paying out of pocket. This can play a role in helping keep
costs down.

There are two big groups opposed to catastrophic care: some liberal
intelectuals and many deluded individuals. The individuals are deluded in that
they believe that they get less when they get catastrophic care. In some ways,
that is true: if given the choice for a full-coverage policy or catastrophic at
the same cost, everyone should prefer full-coverage. However, once the costs
are taken into account, this is not so clear.\FIXME{Check claims that it is
possible to always save.}

In 2013, the results of a large-scale study in Oregon showed only modest health
gains from giving people access to Medicaid.\footnote{The largest medical gain
was on mild depression. This is a suspicious finding diagnostic relies heavily
on self-report of mood and feelings. The psychology literature has several
examples where you can get very different answers from people to this sort of
questionnaire by ``priming'' them first. Being reminded that they won or lost a
lottery to be on Medicaid is exactly the sort of priming that we should be
suspicious of.} However, there were \emph{large gains} in the financial
security of individuals.

\subsection{Healthcare Politics Thoughts}

\thought The individual is not always rationally self-interested, but the
political system is not rationally self-disinterested either.

\thought The US health care system may be many things, but free-market it is
not.

\thought Mandates don't stay modest.\footnote{This is a frequent refrain of
Tyler Cowen in his popular blog \emph{Marginal Revolution}.}

\thought Over the short term, mandates hurt insurance companies; over the long
term, mandates benefit them immensily.

\thought Drug regulation can make two kinds of mistakes: approving when it
should not and failing to approve when it should. We need to watch out against
\textbf{both} of them.


