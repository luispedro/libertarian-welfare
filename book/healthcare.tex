\chapter{Health Care}
\label{chpt:health}

Much has been written about healthcare in the United States. However,
strangely, there has been little thought given to how to expand the amount of
health care provided. As the wonks say, the discussion has all been
\emph{demand side} (who should pay? how? when?) and not \emph{supply side} (who
should provide services? how?). This chapter will discuss how removing
obstacles to the provision of health care can result in cheaper healthcare, by
sacrificing the frills and using technology so that medical quality can be
maintained.

Why is health care so expensive in the United States?

Most other industries are dominated by \emph{almost as good and so much
cheaper} competition. Health care is dominated by \emph{just slightly better
and so much more expensive} competition. We need to focus on the first kind of
competition. To the extent that the extra benefits are not medical, but rather
luxury (plus waiting rooms) or feel-good (over educated doctors taking care of
routine issues), we should be happy to see them go and be replaced by cheaper
medical services.

There are two types of consumers of healthcare: the healthy, looking for
wellness services or with an acute problem which will be cured, and the
chronically sick. The needs of these two groups are very different and we have
to be careful not to sacrifice one to the second.

Low deductibles certainly benefit the healthy, but the value of the deductible
is not a big consideration for a chronically sick individual; while the overall
cost is (and high deductibles can arguably lower the overall cost).

\section{Is the US healthcare system ``free market''?}

No, and the idea that it is is one of the most unhelpful ideas in the whole
debate. When I say that the system is not free market, I do not mean that the
system is not a perfect free, idealized free market. I mean that the healthcare
sector is one of the most heavily regulated, subsidized, and heavy-handed
managed in the United States. Therefore, to argue that its many failings (and
there are many failings, for sure) somehow show that the free market cannot
provide good healthcare is borderline ridiculous.

In fact, the US government \textbf{spends more on healthcare than other
\emph{so called socialist} healthcare systems}. This is without even counting
the largest component of government spending on healthcare: the tax deduction
on employer provided healthcare. If you add this component, then \emph{the US
government spends more on healthcare than any other developed country}. This is
not the hallmark of a free market. % FIXME Add numbers. Tables

The problem is not that the government is not involved, the problem is that it
is involved in number of very negative ways.

In fact, I think that there should be more pressure on free market economists
to explain why the US healthcare system actually works pretty well given this
deep involvement of the state.\footnote{This is partially tongue-in-cheek, but
I think this way to pose the problem has a lot of truth in it. My answer is
that health care is a highly desirable good. Therefore, people are willing to
pay the higher costs for all the extra layers and all the inefficiencies. The
cost of the state involvement does not appear in worse outcomes for most
people, but in higher costs for everyone and reduced access for a minority.}

\subsection{But it is the most free market system, right?}

You might be thinking that ``even if the US is not so free-market, it is more
free-market than other countries.'' No, not really. Many other developed
countries have systems which are arguably more free-market than the US.

Again, other countries have less government spending on healthcare, they will
have fewer impediments to doctors practicing.

\section{Can the government buy healthcare?}

A common argument for heavy government intervention into the healthcare market
is that a market in healthcare is impossible for behavioral reasons: people do
not behave according to rational self-interest when it comes to health and this
makes the resulting market be skewed.\footnote{This argument has a strong
pedigree, it goes back to an article from Nobel laureate Kenneth Arrow,
published in 1963. In a restyled form, it is now very common in the mainstream
press.}

The reason is that health care is different from other markets. While anyone
can easily compare the price of apples or gas and it is something that we buy
on a regular basis, major health purchases are made in a situation of distress.
If they are made for family members, then it is morally impossible for a family
member to perform a cost-benefit analysis. Therefore, as individuals we choose
badly when it comes to health care.

However, it does not immediately follow that the government can do a better
job. We can ask the same question about government deciders: can they make
rational decisions in the context of healthcare? After all, government decisors
are people too and their motivations are not rational self-disinterest. The
political/bureaucratic process has its own behavioral blind-spots. They
naturally differ from the blind spots of the individual, but are they any less
harmful?

For example, in the United States, the FDA had conditionally approved Avastin
for breast cancer. What this means is that it was still experimental, but there
was enough hope that it would have a positive effect in patient's outcomes that
it was OK to let the general patient population have access to the medication
while more studies were being conducted to confirm. Unfortunately, it turned
out that \emph{Avastin is actively harmful for breast cancer
patients}~\cite{avastin-harfmul}\footnote{See this note in Science magazine:
\url{http://news.sciencemag.org/scienceinsider/2011/06/breast-cancer-drug-gets-a-unanimous.html}}.
The problem is that it seems that Avastin has no effects on the cancer, but it
has strong negative side-effects.

This led to a strong outcry. Those outcrying were wrong, of course, but this
fact did nothing to diminish their emotional appeal. Cancer survivors who had
taken Avastin appeared at televised hearings in tears (of course, thanks to
\textbf{other} drugs, many women now, thankfully, survive breast cancer; but
this is unrelated to Avastin).\footnote{See this note in the HufftingtonPost:
`` 'It's saved my life,' said a tearful Sue Boyce, 54, of Chicago.'',
\url{http://www.huffingtonpost.com/2011/11/18/avastin-breast-cancer-fda-approval-revoked_n_1101468.html}.}
Families of current patients and the patients themselves spoke about being
scared this drug will be taken away. The political system quickly responded and
provided guarantees that Medicare would continue to cover (pay for) Avastin.
Avastin costs 100,000 dollars a treatment and it hurts patients, but Medicare
rushed in to provide it at taxpayer expense.

This is not just one isolated incident. As we discuss in more detail below, the
problem is more profound: every time a government regulator stops a good
solution, the victims are invisible or seem unconnected to the regulation; so
there is little penalty in making this sort of mistake. On the other hand, if
it ever allows anything that is later revealed to have been a mistake or even
possibly a mistake, then there is an outcry. Therefore, \emph{government
regulators should be expected to be very risk adverse}.

A separate issue with government regulation (particularly in the American
political system), is that it responds strongly to well-organized pressure.
This leads it to cave to interest groups that manage to garner both money and
sympathy. This is particularly true if it is regulating what others must spend.
Here are a few other completely unscientific treatments that the political
system has required be covered by private health insurance:

\begin{itemize}
\item dolphin treatment
\item homeopathy?
\end{itemize}

Tyler Cowen likes to point out that ``mandates don't stay
modest.''\footnote{This implies an argument for public provision which I find
strong: the government as a provider has a stronger interest not to provide
overly expensive treatments, while as a regulator, it can mandate them as the
political/media system suffers from a blind-spot in this area.}

In a similar vein, can the political system sustain a system of long-term
innovation or will it fall into the temptation of killing the goose that lays
the golden eggs? The FDA already errs on the side of too much caution and not
enough innovation.

\subsection{Should it be easier to practice medicine?}

This is one of those issues that instinctively leads to false dichotomies. I am
not proposing that anyone should be allowed to perform open heart surgery or
that we should have no barrier to access to medical education. Think, however,
of the premed students who almost made it to medical school, but not quite.
They were surely ``doctor material.''

\subsection{Does the FDA save or lose lives?}

Obviously, it does both. We want to make it do more of the saving and less of
the losing, however.

A lot of innovation follows a familiar pattern, the first versions of anything
do not work very well, but are much cheaper. It's \emph{90\% of the effect and
10\% the cost}. Then, as the technology improves, the quality of the new
techniques will often not only match, but overcome what was possible with the
older technology. In medicine, most innovation, however, is \emph{10\% better
at 10 times the cost}. These are not just the musings of extreme
anarcho-capitalists. Michael Mandel at the Progressive Policy Institute has
made exactly these points in a brief on healthcare
(\url{http://autismcrisis.blogspot.pt/2007/08/autism-and-aba-in-uk-controlled-trial.html}).

Still today, Vioxx is often brought up as a scandal and a failure of
regulation. However, Vioxx was not that awful: all pain drugs have terrible
side-effects, Vioxx was oversold as not having any, which was not true, but it
is no worse than any of the alternatives.\footnote{Still today, there are
arguments that \emph{Vioxx should be put back on the market} as some patients
might be better off managing the risks associated with it rather than go
without it. This is tangencial to my point, but I want to point out that the
idea can be defended.} Furthermore, even if we accept that Vioxx should not
have been sold, we need to be careful not to over-react to any mistake that is
only found a posteriori.

\subsection{Healthcare Politics Thoughts}

\thought The individual is not always rationally self-interested, but the
political system is not rationally self-disinterested either.

\thought Mandates don't stay modest.

\thought Drug regulation can make two kinds of mistakes: approving when it
should not and failing to approve when it should. We need to watch out against
\textbf{both} of them.


