\chapter{Introduction}

That there is a role for the state in guaranteeing everyone a set of basic
goods so that they may fully participate in society as equal citizens, will not
be questioned. However, the way in which state institutions go about performing
their tasks does not always respect the recipients of such aid.

There are a few interlocked goals to this book. One is to save the economic
freedom arguments from those who have used them (and often abused them) from
the conservative side.\footnote{For example, conservative activists will often
use the language of property rights to defend zoning laws that keep poor,
dark-skinned, people away from rich people. However, zoning laws are very much
anti-free-market. See
\url{http://www.theatlantic.com/business/archive/2012/03/affordable-housing-and-social-engineering-in-new-jersey/255269/} for a perfect example.} % FIXME: inline the comment

The other is to present free-market solutions to social problems not only as
more efficient than current alternatives (which is often the angle that is
taken), but \emph{as morally superior}. Not morally superior only because they
involve less coercion or better respect property rights, but morally superior
because of the way in which recipients are treated.

The free market is a friend, not a foe of the downtrodden.


