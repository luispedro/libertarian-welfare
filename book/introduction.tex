\chapter{Introduction}

Despite all the rhetorical polarization in modern day America, there is central
ground on which both the Democrats and Republicans agree. The Democrats are not
socialists (they will not completely nationalize industry and hand it to
politically apointed technocrats) and the Republicans will not eviscerate the
institutions of the welfare state (in fact, the last time that Republicans
controlled both the Congress and the Presidency, they expanded the welfare
state, in the form of Medicare Part D). No matter who wins the next election,
the American government will be about as big as it is now (at most a few
percentage points bigger or smaller).

The question is not whether to have state or not, nor even whether to have a
welfare state or not, but which forms it should take. This book will be part of
this conversation.

There is a role for the state in guaranteeing everyone a set of basic goods so
that they may fully participate in society as equal citizens. This book will
not question this premise. However, the way in which state institutions perform
their tasks does not always respect the recipients of such aid. The usual
charge is that they breed dependency or complacency. I will also argue that
they insult its recipients.

\bigskip

The welfare state is a big subject. I will focus on a few subjects: education,
health care, basic income provisions, and the environment (which is, strictly
speaking, not a welfare state topic, but it fits the rest of the book). I will
discuss how American institutions currently operate and present alternatives.
These will not be utopian alternatives, but we will see examples from other
societies. In particular, a few societies will keep coming up, the S-countries:
Singapore, Sweden, and Switzerland. Neither is a perfectly good model for the
US to follow. Singapore is a dictatorship and, no matter how mild of a
dictatorship, its political system is abhorrent. Still, we can look at some of
their economic policies as examples. Sweden and Switzerland are both
democracies, but are small countries, with a culturally and racially
homogeneous population (very different from the US). Still, we can learn from
their examples. Denmark and Finnland, even though they do not start with an S,
will show up as well. The Nordic countries\footnote{I will keep Norway out of
this, not because there is anything wrong with its economic system or culture,
but because there is not much to be learned from an oil-rich state.} are often
darlings of the Left, but they could as well be darlings of the free-market
Right.

It may be the case that a complete rewrite of the system to a Universal Basic
Income as favoured, for example, by \citet{Murray_IOH}, is a much better
alternative. However, my focus here is, to borrow a phrase from the blog
\emph{Marginal Revolution}, ``small steps to a much better world.''

It is not a partisan book. I hope that both Democrats and Republicans can learn
something from it and see a middle ground.

\bigskip

There are a few interlocked goals to this book. One is to save the economic
freedom arguments from those who have used them (and often abused them) from
the conservative side. For example, conservative activists will often use the
language of property rights to defend zoning laws that keep poor, dark-skinned,
people away from rich people. However, zoning laws are very much
anti-free-market.\footnote{See
\url{http://www.theatlantic.com/business/archive/2012/03/affordable-housing-and-social-engineering-in-new-jersey/255269/}
for a perfect example. Matt Yglesias makes the same point in his wonderful book
``The Rent is Too Damn High.''} % FIXME: inline the comment

The other is to present free-market solutions to social problems not only as
more efficient than current alternatives (which is often the angle that is
taken in defending them), but \emph{as morally superior}. Not morally superior
only because they involve less coercion or better respect property rights (the
traditional libertarian goals), but morally superior because of the way in
which recipients are treated. At the interface of the welfare state and the
recipients, there are no magic ``rights,'' but a real interaction between
recipients and bureacracies. The alternative models I propose here will empower
(in the literal sense of the word, will give power to) the recipients.

A family who approaches a school system with a voucher that gives them choices
is more empowered and will get more respect than a family who, because they
cannot afford a private school, approaches their failing neighborhood school.

There is always almost complete choice for parents who can afford it. They can
resort to private schools or simply buy a house in their desired school
district. The discussion is only whether that choice should be extended to not
so fortunate parents. We should not lose track of this simple fact.  In fact,
if the current system was a voucher system, then I do not think that there
would be very strong arguments against moving to a limited-choice system
(currently, in the US, we can find systems ranging from voucher to assigned
school as well as many systems in between where there is a certain amount of
choice, but not complete choice).

In Chapter~\ref{chpt:education}, I will discuss the evidence on school choice
and whether children benefit from switching schools under that system. I think
that there is evidence for a small positive effect, but \emph{even if there was
no effect on grades or other outcomes, school choice would be the more moral
system because it treats recipients with greater respect}.

In other chapters, I will reverse the focus and look at provision of services.
Too often the focus of welfare discussion is on the demand side. That is, how
to pay for it? This is most evident in health care. Almost all of the
discussion on health care reform in 2008--10 was about how to pay and who
should pay when, given a fixed set of resources.  In the health care chapter,
we will look at the other side of the ledger, how to provide services and I
will argue that health care costs are so high, at least in part, due to
over-regulation in the provision of services. If it was easier for doctors
trained abroad (including Americans who attended medical schools in other
countries) to practice in the US, if it was easier to provide medical services
in innovative ways; then health care could be cheaper for everyone. But
over-acreditalization (demanding that even subsidiary staff have advanced
degrees) and too many barriers to innovation have driven up costs. Again, we
need not see the alternatives as either no-regulation or our current system.
Every regulation has costs and benefits and the US has too often taken the view
that ``no cost is high enough.''


