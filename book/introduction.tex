\chapter{Introduction}

There is a role for the state in guaranteeing everyone a set of basic goods so
that they may fully participate in society as equal citizens. However, the way
in which state institutions go about performing their tasks does not always
respect the recipients of such aid.

Despite all the rhetorical polarization in modern day America, there is big
center on which both the Democrats and Republicans agree. The Democrats are not
socialists (they will not completely nationalize industry) and the Republicans
will not eviscerate the institutions of the welfare state. No matter who wins
the next election, the American government will be about as big as it is now
(at most a few percentage points bigger or smaller).

The question is not whether to have state or not, nor even whether to have a
welfare state or not, but which forms it should take.

There are a few interlocked goals to this book. One is to save the economic
freedom arguments from those who have used them (and often abused them) from
the conservative side.\footnote{For example, conservative activists will often
use the language of property rights to defend zoning laws that keep poor,
dark-skinned, people away from rich people. However, zoning laws are very much
anti-free-market. See
\url{http://www.theatlantic.com/business/archive/2012/03/affordable-housing-and-social-engineering-in-new-jersey/255269/} for a perfect example.} % FIXME: inline the comment

The other is to present free-market solutions to social problems not only as
more efficient than current alternatives (which is often the angle that is
taken), but \emph{as morally superior}. Not morally superior only because they
involve less coercion or better respect property rights (the traditional
libertarian goals), but morally superior because of the way in which recipients
are treated. At the interface of the welfare state and the recipients, there
are no magic ``rights,'' but a real interaction between recipients and
bureacracies. The alternative models I propose here will empower (in the
literal sense of the word, will give power to) the recipients.

A family who approaches a school system with a voucher that gives them choices
is more empowered and will get more respect than a family who, because they
cannot afford a private school, approaches their failing neighborhood school.
In Chapter~\ref{chpt:education}, I will discuss the evidence on school choice
and whether children benefit from switching schools under that system. I think
that there is evidence for a small positive effect, but \emph{even if there was
no effect on grades or other outcomes, school choice would be the more moral
system because it treats recipients with greater respect}. In fact, if the
current system was a voucher system, then I do not think that there would be
very strong arguments against moving to a limited-choice system (currently, in
the US, we can find systems ranging from voucher to assigned school as well as
many systems in between where there is a certain amount of choice, but not
complete choice\footnote{There is always almost complete choice for parents who
can afford it. They can resort to private schools or simply buy a house in
their desired school district. The discussion is only whether that choice
should be extended to not so fortunate parents. We should not lose track of
this simple fact.}).

Too often the focus of welfare discussion is on the demand side, how to pay for
it? This is most evident in health care. Almost all of the discussion on health
care reform in 2008--10 was about how to pay, given a fixed set of resources.
In the health care chapter, we will look at the other side of the ledger, how
to provide services and I will argue that health care costs are so high, at
least in part, due to over-regulation in the provision of services.

The welfare state is a big subject. I will focus on a few subjects: education,
health care, basic income provisions, and the environment. I will discuss how
American institutions currently operate and present alternatives. These will
not be utopian alternatives, but we will see examples from other societies. In
particular, a few societies will keep coming up, the S-countries: Singapore,
Sweden, and Switzerland. Neither is a good model for the US to follow.
Singapore is a dictatorship and, no matter how mild of a dictatorship, its
political system is abhorrent. Still, we can look at some of their economic
policies as examples. Sweden and Switzerland are both democracies, but are
small countries, with a culturally and racially homogeneous population.

It may be the case that a complete rewrite of the system to a Universal Basic
Income as favoured, for example, by \citet{MurrayUBI}, is a much better
alternative. However, my focus here is, to borrow a phrase from the blog
\emph{marginal revolution}, ``small steps to a much better world.''

The free market is a friend, not a foe of the downtrodden.


