\chapter{International Comparisons}\label{chpt:international}

But wait a minute, I thought that libertarians thought that Europe was a
socialist basket case? Maybe some do. They are wrong.

Europe is not a single system, but more like what some Republicans want for the
US: a weak central government in Brussels and strong states.\footnote{It is
noteworthy that American States have certain powers that States in Europe have
already lost to the central government of the EU. This is often very surprising
to Europeans.} The member-states of the EU have their own cultures, their own
histories, and their own institutions. Some are very free-market and very
successful. Others are neither. And yet a third group is in between.

Which are these very-successful European free-market economies? Start at the
top: Denmark, then Sweden, Switzerland, Finland, Holland, and Ireland.
Luxembourg also makes the cut, but it is arguably very small.

I can hear some readers cry out \emph{Sweden? Are you crazy? They are the land
of high taxes. They are quasi-socialism, the true third way.}\footnote{I can
even feel some readers angrily swiping away this book and choosing something
else from their e-book library.} I think these readers may have a mostly
correct, but very out-of-date picture of Sweden. Modern Sweden's economic
history can be resumed into three phases:

\begin{enumerate}
\item Free-trade, free-market capitalism: (late 1800s--late 1930s). Sweden went
from one of the poorest to one of the richest countries in Europe. Noticeably,
this happened without a lot of inequality. This is one of the great free-market
success stories.
\item Social democracy: (1950s-late 1980s). I skipped the war period, even
though the social-democrats were already in power. If a major world war is
going on, it hardly feels fair to blame the economic policies for all the
hardship that Swedes endured during the war. In the early 1950s, when the
damages of the war to the physical infrastructure had been repaired, Sweden was
one of the richest countries in the world. Government took up 21\% of GDP.
Then, government started growing by 1\% a year. Sweden stagnated economically.
It was still a very pleasant country to live in,  but it was not growing as
before. This is when Sweden got its reputation of a successful third-way
country. However, it had had a major head start and was losing ground.
\item Return to capitalism (1990s--present). Modern Sweden has free-market
social security, a school-system based on vouchers, no capital gains taxes, no
inheritance taxes. Taxes are still very high, but on a downward trend, and the
tax system is very simple: most people do not even file taxes. Instead, you
receive a tax estimate from the government based on the filings of your
employer and you have the option of just accepting it without any major hassle.
In the United States, the complexity of the tax system is itself a significant
tax.
\end{enumerate}

Part of the goal of this book is to reclaim the Nordic countries for the
free-market column.

