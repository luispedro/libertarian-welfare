\chapter{International Comparisons}\label{chpt:international}

The United States is certainly the country where freedom (or Freedom, with a
capital letter) is a bigger part of the political discourse. Throughout the
second half of the 20th century, it was also one of the countries that most
resisted the left-wing shift in economics that swept through Europe. Thus, most
people reason, it must certainly be the most free market developed country. In
fact, many discussions of free-market are a proxy fight for \emph{the American
system}.

\begin{table}
\centering
\begin{tabular}{lrr}
\toprule
Country & Points & Rank \\
\midrule
Hong Kong             & 89.3 &  1 \\
Singapore             & 88.0 &  2 \\
Australia             & 82.6 &  3 \\
New Zealand           & 81.4 &  4 \\
Switzerland           & 81.0 &  5 \\
Canada                & 79.4 &  6 \\
Chile                 & 79.0 &  7 \\
Mauritius             & 76.9 &  8 \\
Denmark               & 76.1 &  9 \\
\emph{United States}  & 76.0 & 10 \\
Ireland               & 75.7 & 11 \\
Bahrain               & 75.5 & 12 \\
Estonia               & 75.3 & 13 \\
United Kingdom        & 74.8 & 14 \\
Luxembourg            & 74.2 & 15 \\
\bottomrule
\end{tabular}
\caption{Heritage Ranking of Economic Freedom for 2013. This data is available
online at \url{http://www.heritage.org/index/ranking}.}
\label{tab:heritage}
\end{table}

\begin{table}
\centering
\begin{tabular}{lr}
\toprule
Country & Rank \\
\midrule
Singapore & 1 \\
Hong Kong SAR, China & 2 \\
New Zealand & 3 \\
\emph{United States} & 4 \\
Denmark & 5 \\
Norway & 6 \\
United Kingdom & 7 \\
Korea, Rep. & 8 \\
Georgia & 9 \\
Australia & 10 \\
Finland & 11 \\
Malaysia & 12 \\
Sweden & 13 \\
Iceland & 14 \\
Ireland & 15 \\
\bottomrule
\end{tabular}
\caption{World Bank's ``Ease of Doing Business'' index (2012). This data is
available online at \url{http://www.doingbusiness.org/}.}
\label{tab:ease-of-doing-business}
\end{table}

Throughout the book, I have often mentioned in a positive light several
countries, mostly the S-countries: Singapore, Sweden, Switzerland. Also,
Denmark, Holland, Finland. Many of these European countries are excellent
models for the United States (even more than Singapore, which, arguably, is
culturally very different). But wait a minute, I thought that libertarians
thought that Europe was a socialist basket case? Maybe some do. They are wrong.

Europe is not a single system, but more like what some Republicans want for the
US: a weak central government in Brussels and strong states.\footnote{It is
noteworthy that American States have retained certain powers that States in
Europe have already lost to the central government of the EU. This is often
very surprising to Europeans.} The member-states of the EU have their own
cultures, their own histories, and their own institutions. Some are very
free-market and very successful. Others are neither. And yet a third group is
in between.

Look at Tables~\ref{tab:heritage} and~ref{tab:ease-of-doing-business}. The
first table is compiled by the conservative Heritage Foundation and attempts to
measure economic freedom. These indices are not perfect, but Canada and Denmark
both outdo the United States. The very conservative Heritage Foundation would
like the United States to be more like Canada.\footnote{I have to point out
that I completely disagree with the Heritage Foundation's views on non-economic
matters. There, they toss their libertarianism out the window and want strong
government intervention, be it in other countries (they support the ``War on
Terror'' of Presidents Bush and Obama) or cheering for law enforcement officers
(which is often abusive towards poorer folks, who will often suffer from being
both victims of crime and abused by police).}

It is interesting that in both these tables, New Zealand is ahead of the United
States in economic freedom. A 2012~book by David Hackett Fischer\footnote{The
book is called \emph{Fairness and Freedom: A History of Two Open Societies: New
Zealand and the United States}. I found it a very interesting history of New
Zealand even if the fairness vs.\ freedom thread was often forced on actual
events rather than a natural reading of them.} argued that the two societies
differed mostly in that New Zealand's culture was centered on fairness and
equity, whilst the US was centered on liberty and freedom (all of these
concepts should be read in their widest possible sense). This was often
interpreted as a defense of left-wing policies, but we find that a focus on
fairness lead New Zealand to a more free-market system than the American focus
on freedom. Whenever liberal history professors and the conservative Heritage
Foundation agree with each other and beneath the apparent disagreement, point
at New Zealand as an example; whenever that happens, we should pay attention.

Sweden clocks in at number 18, just behind Finland and the Netherlands. The low
rating of Sweden is something that will be likely improving over the next few
years. Sweden is an excellent model. I can hear some readers cry out
\emph{Sweden? Are you crazy? They are the land of high taxes. They are
quasi-socialism, the true third way between capitalism and
socialism.}\footnote{I can even feel some readers angrily swiping away this
book and choosing something else from their e-book library.} I think these
readers may have a mostly correct, but very out-of-date, picture of Sweden.
Modern Sweden's economic history can be resumed into three phases:

\begin{enumerate}
\item Free-trade, free-market capitalism: (late 1800s--late 1930s). Sweden went
from one of the poorest to one of the richest countries in Europe. Noticeably,
this happened without a lot of inequality. This is one of the great free-market
success stories.
\item Social democracy: (1950s-late 1980s). I skipped the war period, even
though the social-democrats were already in power. If a major world war is
going on, it hardly feels fair to blame the economic policies of the government
in power for all the hardship that Swedes endured during the war (we should
blame the war for that). In the early 1950s, when the damages of the war to the
physical infrastructure had been repaired, Sweden was again one of the richest
countries in the world. Government took up 21\% of GDP. Then, government
started growing by 1\% a year. Sweden stagnated economically.  It was still a
very pleasant country to live in,  but it was not growing as before. This is
when Sweden got its reputation of a successful third-way country. However, it
had had a major head start and was losing ground.
\item Return to capitalism (1990s--present). Modern Sweden has free-market
social security, a school-system based on vouchers, no capital gains taxes, no
inheritance taxes. Taxes are still very high, but on a downward trend, and the
tax system is very simple: most people do not even file taxes. Instead, you
receive a tax estimate from the government based on the filings of your
employer and you have the option of just accepting it without any major hassle.
In the United States, the complexity of the tax system is itself a significant
tax.
\end{enumerate}

The Nordic countries are well-functioning societies and we can learn from
them.\footnote{Needless to say, not everything they do is good and some of the
good things only work so well in culturally homogeneous small countries.}
Normally this argument comes from the left, but part of the goal of this book
is to reclaim them for the free-market defenders. The conservative right-wing
will still find much to disagree with when it comes to, for example, the
limited role that religion plays in these societies and the legal acceptance of
gay marriage and the ease of divorce.\footnote{Although, for all their legal
liberalism, Swedes behave like conservatives: a larger percentage of children
grow up with both biological parents in Sweden than in the supposedly more
conservative United States. This is even trumpeted by the Swedish government in
their webpage of information on Sweden, where emphasis is given to the fact
that ``Some 90 percent of children spend their early years living with both
parents.''\backnote{\url{http://www.sweden.se/eng/Home/Society/Child-care/Facts/Children-in-Sweden}}.
If you measure throughout the whole childhood, 69\% of children grow up in
intact families to age 18 in Sweden, while the figure for the United States is
56\%\backnote{see ``Family Structure and Child Outcomes in the United States
and Sweden'' by A. Björklund, D. K. Ginther, and M. Sundström,
\url{ftp://ftp.iza.org/dps/dp1259.pdf}.}}

The United States does have a smaller state, but only barely. Government
spending is at about 40\% of GDP, just below Canada, and slightly above
Luxembourg, for example.\footnote{One of my pet peeves is when someone arguing
for more government in the US, points out that the Federal government is only
20\% of GDP, much less than in other countries. This neglects that, in the US,
half the spending is at a more local level, be it state, city, county, or any
other of these levels. I could just as well say that Denmark has little
government spending because so much of it is at a sub-national level.} I think
this understates the case, however, as there are many ``tax expenditures'' in
the United States, more so than in other countries. A tax expenditure is what
happens when instead of subsidizing activity A, the government declares that
activity A is partially or fully exempt from taxes. This has the advantage that
it allows politicians (especially pseudo-small government ones) to claim they
reduced the tax bill. However, it is very much unlike reducing taxes if it only
applies to the favored constituency.

Part of the agenda of this book is to change the perception of these countries.

\section{In Development}

Perhaps the biggest change that ``libertarian welfare'' can make is in the
developing world. Institutions (formal and informal) in developing countries
tend to be bad-to-awful (there are exceptions, naturally).

Conditional cash transfers are already a big weapon in poverty alleviation.
This is an idea that transitioned from the libertarian right to being accepted
by the liberal left (a path I hope for many of the other ideas in this book).

Unconditional cash transfers will be perhaps the purest expression of the idea
that the major problem with poverty is lack of resources and that you should
trust people over institutions.\footnote{This is true of both national and
international institutions.} \FIXME{Reference Chris Blattman's RCT here}
\FIXME{Reference Give Directly}

\section{International Comparison Thoughts}

\thought Compare policies not speeches.

\thought Compare policies not stereotypes.

\thought Many European countries are arguably more free-market than the United
States.

