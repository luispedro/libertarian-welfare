\chapter{Environment}
\label{chpt:environment}

The environment is not, strictly speaking, a welfare state topic. However,
bringing up environmental concerns is a common objection to small government
(for many, it is \emph{the} objection to smaller government).

However, this is a topic in which small government advocates should have the
upper hand. Markets are great at optimizing resource utilization if we let the
price system reflect all costs (including environmental costs).

\section{Subsidies for Clean Energy}

According to the International Energy Association, subsidies for fossil fuel
extraction around the world are larger than the subsidies for clean energy.
Just eliminating these subsidies could be a big boost in preventing climate
change.\footnote{``Fatih Birol, chief
economist at the International Energy Agency, said such a move could
provide half of the carbon savings needed to stop dangerous levels of climate
change.'' (in the British newspaper, The Guardian).%
% Link http://www.guardian.co.uk/environment/2012/jan/19/fossil-fuel-subsidies-carbon-target
} Therefore, in a free-market without any subsidies either way (neither for
fossil nor for clean energy), we would expect to see less fossil fuel use.

Naturally, some of these subsidies might serve other worthwhile purposes. For
example, if you live in a cold climate and low-income families in your area can
apply for heating aid in the winter in the form of lowered gas prices. This is
a subsidy to pollute, yes, but it serves another important purpose: to make
sure that low-income families (including children in those households), do not
suffer too much due to the cold. This is a subsidy for fossil consumption that
many would the loath to cut.

% http://www.washingtonpost.com/blogs/wonkblog/post/why-775-billion-in-fossil-fuel-subsidies-are-hardto-scrap/2012/06/18/gJQABaQUlV_blog.html
% http://www.forbes.com/sites/timworstall/2012/09/17/must-we-again-about-fossil-fuel-subsidies-around-the-world/
