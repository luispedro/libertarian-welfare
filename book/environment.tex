\chapter{Environment}
\label{chpt:environment}

The environment is not, strictly speaking, a welfare state topic. However,
bringing up environmental concerns is a common objection to small government.
For many, it is \emph{the} objection to smaller government.

However, this is a topic in which small government advocates should have the
upper hand. Markets are great at optimizing resource utilization if we let the
price system reflect all costs (including environmental costs).

\subsection{The Acid Rain Success Story}

I am old enough to remember the fear that acid rain would kill off all lakes
and rivers. While some of this was fear mongering to sell newspapers.

\section{Subsidies for Clean Energy}

According to the International Energy Association, subsidies for fossil fuel
extraction around the world are larger than the subsidies for clean energy.
Just eliminating these subsidies could be a big boost in preventing climate
change.\footnote{``Fatih Birol, chief
economist at the International Energy Agency, said such a move could
provide half of the carbon savings needed to stop dangerous levels of climate
change.'' (in the British newspaper, The Guardian).%
% Link http://www.guardian.co.uk/environment/2012/jan/19/fossil-fuel-subsidies-carbon-target
} Therefore, in a free-market without any subsidies either way (neither for
fossil nor for clean energy), we would expect to see less fossil fuel use.

Naturally, some of these subsidies might serve other worthwhile purposes. For
example, if you live in a cold climate and low-income families in your area can
apply for heating aid in the winter in the form of lowered gas prices. This is
a subsidy to pollute, yes, but it serves another important purpose: to make
sure that low-income families (including children in those households), do not
suffer too much due to the cold. This is a subsidy for fossil consumption that
many would the loath to cut.

% http://www.washingtonpost.com/blogs/wonkblog/post/why-775-billion-in-fossil-fuel-subsidies-are-hardto-scrap/2012/06/18/gJQABaQUlV_blog.html
% http://www.forbes.com/sites/timworstall/2012/09/17/must-we-again-about-fossil-fuel-subsidies-around-the-world/

\section{Pollution Taxes}

A tax is good when it achieves two purposes: (1) raises revenue and (2) curbs
harmful behavior.

Taxes do not have strictly revenue raising effects. A tax on any activity will
always curb that activity. A tax on labor causes people to hire others
less. A tax on investment gains may curb investment.\footnote{The true effects
are much more complex. Who ends up really paying for a tax is complex.
Moreover, a tax on labor makes workers poorer, which may encourage them to
actually try harder to work.} A good tax is that which, if it fails to raise
revenue because people curbed that activity, will still be hailed a success.

Pollution taxes are exactly such good taxes. If the result is that millions are
invested in ``tax avoidance'' by employing less polluting technologies and very
little revenue is obtained, this will be a major success.

Many small government advocates would agree with everything above, but not
actually trust that the government would fall through with the second half of
the bargain: reducing other taxes. In their view, this line of argument could
be a bait-and-switch on the part of big government liberals.\footnote{Many
thoughtful liberals feel the same way about free trade. They agree that there
is a potential for everyone to be better off, but do not trust that the
government is very effective at redistributing the gains to the losers of free
trade.}

\subsection{Environmental Thoughts}

``Pollution is over-regulated, but under-priced.''

``A good tax both raises revenue and curbs harmful behavior.''

