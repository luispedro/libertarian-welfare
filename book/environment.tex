\chapter{Environment}
\label{chpt:environment}

Strictly speaking, the environment is not a part of the welfare state, as the
welfare state is traditionally understood. However, this traditional
understanding of the welfare state is over 200 years old. It predates
environmental awareness as it was born when getting enough to eat was still a
major concern of most of the population and caring about the welfare of animals
was for the rich gentry.

Today, however, we are thankfully able to broaden our horizons to include the
environment as a concern. For many people, environmental concerns are the most
important objection to smaller government.

This is due more to historical contingencies and sociological coincidences than
logical thinking. \emph{The environment is a topic in which free market
advocates should have the upper hand.} Markets are great at optimizing resource
utilization if we let the price system reflect all costs (including
environmental costs).

Most thoughtful free-market intelectuals accept these permises and support a
carbon-tax or cap-and-trade system on CO$_2$ emissions like the one that was so
successful in curbing acid rain, as I detail in the next section.

Unfortunately, the pro-business lobby has often coopted the free-market
thinkers in this matter (as in others). The pro-business lobbies will use
free-market rethoric when it is not even very applicable (and then turn around
and ask for subsidies). Because of tactical alliances, the free-market movement
has let itself be coopted and this has led to a tolerance for the most absurd
global warming denialism and ``us vs.\ them'' thinking.\footnote{It is also
true that many on left have used environmentalism as a weapon against
free-markets even when they do not really care about it. I remember once, many
years ago, arguing with a liberal so convincingly that global warming was best
solved by more markets that he started arguing for the global warming skeptic
position (``because we are not so sure that warming is happening, this should
not be a reason to start privatizing CO${}_2$ emissions''). This was many years
ago, before global warming was such a divise issue. It would no longer be
possible today, I think.}

We need to stop accepting the idea that solving environmental problems requires
strong centralized command-and-control. In fact, command-and-control
environmentalism is not always as effective as the free-market environmentalism
at achieving its environmental goals.

Pollution seems to follow an inverted U evolution. Very poor societies have
little industrial pollution (they often have other sorts of pollution, though,
such as animal waste, which is a major source of human disease). As they get
richer, industrial pollution starts to grow. The choice is made to pay the
costs of pollution for the increase in living standards. At some point,
however, the choices start to be made in the other direction. Pollution peaks
and starts declining (China might now be in this transition). When we are rich
(the Western world), we can afford to increasingly take the environmental
impacts into consideration and spend some of our wealth on the environment.
While in the short term, there is a tradeoff between economic growth and the
environment, in the long term, there is none.

\section{Externalities}

Environmental problems are what economists call \emph{externalities}. This
means that certain business dealings impact external people. For example, a
company produces a toy cheaply by dumping heavy metals in the river and I buy
it, then both parties to the transaction (me and the company selling me the
toy) are better off. My daughter, who will most likely receive and play with
the toy, is also better off. However, there are external parties who lose:
those people who live near the river, anyone who would like to eat the fish
without harm and everyone who enjoys a clean river. This, of course, is a
horrible situation and needs to be rectified.

Where common reasoning goes wrong is to then assume that the alternative is
command-and-control by politically appointed bureacrats. This model, whereby
there is a constant negociation between polluters and the regulatory agencies
as to what is an acceptable amount of pollution does lead to a reduction in
pollution but does so in an overly costly manner, which ultimately means more
pollution than we would otherwise get.

For very toxic substances, the solution might be an outright ban. For example,
the neurological effects of lead seem to be damaging enough that we want to ban
leaded gasoline and paint. They are unsafe in even small amounts (and for very
small benefits). Lead has even been recently linked to the rise and subsequent
fall of urban crime in the United States and elsewhere (the US may be ahead of
the curve as it moved early to ban leaded gasoline,
\href{http://www.motherjones.com/kevin-drum/2013/01/lead-and-crime-ill-be-melissa-harris-perry-show-sunday-10-am}{Kevin
Drum predicts} that Latin America will likewise benefit from a drop in violent
crime over the next few decades).\backnote{See
\url{http://www.motherjones.com/environment/2013/01/lead-crime-link-gasoline}
}

\subsection{The Acid Rain Success Story}

I am old enough to remember the fear that acid rain would kill off all lakes
and rivers. While some of this was fear mongering to sell newspapers, there was
a true problem. At the time in the US, Ronald Reagan was President and this led
to the attempt to use markets to combat harmful SO${}_2$ emissions. The ideas
had been around for a while, but this was the first big attempt at applying it
to real world.

The system adopted was called \emph{cap-and-trade}, which roughly runs as
follows: (1) a quantity of SO${}_2$ is defined as the acceptable amount (this
is the \emph{cap}), (2) permits are handed out to organizations, which (3) can
either use them to pollute or sell them (this is the \emph{trade} half of the
scheme).

This way, the democratic/administrative process can decide exactly how much
polutant will be emited. This should be known and set well in advance so that
economic agents may plan. Ideally, the schedule for the next 10 to 20 years
is publicly known in advance. This simplicity has the advantage that it is a
single number and there is less room to hide a few bodies in the details by
giving in to special interests (see the coal story below).

Curiously, it does not matter (environmentally) how the permits are given out.
For fairness, we would ideally auctioned them off 100\%, where everyone gets an
equal chance to get one (in the past, environmental groups have even bought
permits and then let them expire without polluting; which is a perfectly valid
use of the system). At the other extreme, permits can just be given out to
favored organizations. In fact, typically, something in between is adopted:
existing organizations are grandfathered in and receive a number of permits;
while others are auctioned off.

However, in both cases, firms have the same incentives: whenever they pollute
less, they can make an extra profit; or if their competitors figure out how to
produce the same product with less pollution, they will be able to undercut
them. Therefore, they will consider poluting a cost and try to minimize it: the
environmental effects are the same independently of who gets the permits
initially as long as they can sell them on the open market.

The grandfathering aspect simply gives a very valuable piece of paper (or,
rather, electronic permit) to a select few, which is obviously very unfair to
others and impedes new competitors from entering the market (and even existing
competitors from gaining market share). Ideally, permits would be always
auctioned off for these reasons. For public choice reasons, it is often
necessary to offer a certain amount of grandfathering to achieve buy-in from
the existing producers (so that they lobby for cap-and-trade instead of against
it). The fraction that is auctioned off is not fixed and can increase to 100\%
with time so that the grandfathering effect fades away.

An interesting, rarely mentioned, coda to this story is that, emboldened by
this success, the US delegation to the United Nations which discussed climate
change, lobbied strongly in favor of similar market-based solutions to climate
change, often against the opinions of most other countries. However, forceful
negociation by the US made the final result, the so called Kyoto Protocol, a
treaty which catered to its pro-market prejudices. In the end, as is
well-known, the US did not ratify the treaty, while many nations which had long
argued against the pro-market bias of Kyoto, did.

\subsection{The Dirty Coal Failure}

\subsection{Preserve Endangered Species by Having People Shoot Them for Money}

In 2012, King Juan Carlos of Spain, was caught hunting elephants in Botswana.
In fact, the trip would have been a secret, except that the King broke his hip.
As a result of this experience, the Spanish branch of the World Wildlife Fund
(WWF) stripped the King of the title of honorary President.

Partly, the trouble with the trip was that it was unseemly for the head of
state to travel in luxury (as the guest of a wealthy business-man) during a
time of economic crisis. This had nothing to do with the environment, and I
will let the Spanish people comment on that aspect of the controversy.

However, it was also believed that hunting elephants (an endangered species)
was an environmental crime. This was what was behind WWF's decision to ``fire''
the King as Honorary President.

This is wrong. The opposite is true: \emph{by hunting elephants, Juan Carlos
was helping the species and keeping it away from extinction}! This may seem
counter-intuitive, but consider that neither the pig nor the cow are in any
danger of extinction. These species are often hunted down because they are a
nuisance to humans or livestock. Trying to expect people to tolerate them out
of the sake of conservation is often too much to expect. However, when hunters
pay thousands of dollars for a hunting license, the animals are now valuable.
Thus structures are encouraged to preserve them.

Additionally, these schemes obtain the support of the local population. Blanket
prohibitions on hunting will often come across as impositions from mostly-white
foreigners in a continent with a history of such meddling. As such, the illegal
poaching will not be seen by local people as a crime, but as a way of making a
living. However, by letting the locals benefit from limited hunting, poaching
now becomes a crime not only against the laws imposed from above, but also
against the community. Thus, the local community will start to collaborate in
the protection of the species instead of attempting to undermine it. It is also
morally preferable that locals benefit from any protection schemes rather than
only losing out (which is what happens when they are forbidden to hunt and even
grow crops).\FIXME{I think Karol Boudreax's has some relevant results wrt
Namibia}

% Check up what type of elephant Juan Carlos was hunting
% http://www.rnw.nl/africa/bulletin/big-game-hunts-defended-after-spanish-kings-disputed-trip

\subsection{Fisheries Collapse}

While big fauna (and small cute animals) capture the most headlines, beneath
the waves of the oceans, many species are now on verge of extinction. This is
one of the most overlooked environmental tragedies of today. Unlike land based
extinction where the proximal cause is often habitat destruction, the major
problem in the oceans is overfishing.

This is the archetypical tragedy of the commons: everybody can fish and keep
the proceeds. Therefore, the individual incentive is to get as much fish as you
can. The overall result is over-consumption and is bad for everybody.

Again, with this class of problems there are traditionally two possible
solutions: political/bureacratic allocation or privatization.\footnote{There
is a third possibility: to use non-market and non-state mechanisms of
allocation. These were so foreign to conventional economics thinking that
describing them as they exist in the real-world led to a Nobel Prize in
Economics in to Elinor Ostrom and Oliver Wiliamson, in 2008. According to
standard economics, these mechanisms should not work very well; and yet they
do. Unfortunately, these work better in small communities where trust is
wide-spread than in a large society and are easier to describe than to
engineer.}

\section{Subsidies for Dirty Energy}

According to the International Energy Association, subsidies for fossil fuel
extraction around the world are larger than the subsidies for clean energy.
Just eliminating these subsidies could be a big boost in preventing climate
change.\footnote{``Fatih Birol, chief
economist at the International Energy Agency, said such a move could
provide half of the carbon savings needed to stop dangerous levels of climate
change.'' (in the British newspaper, The Guardian).%
% Link http://www.guardian.co.uk/environment/2012/jan/19/fossil-fuel-subsidies-carbon-target
} Therefore, in a free-market without any subsidies either way (neither for
fossil nor for clean energy), we would expect to see less fossil fuel use.

Naturally, some of these subsidies might serve other worthwhile purposes. For
example, if you live in a cold climate and low-income families in your area can
apply for heating aid in the winter in the form of lowered gas prices. This is
a subsidy to pollute, yes, but it serves another important purpose: to make
sure that low-income families (including children in those households), do not
suffer too much due to the cold. This is a subsidy for fossil consumption that
many would the loath to cut.

% http://www.washingtonpost.com/blogs/wonkblog/post/why-775-billion-in-fossil-fuel-subsidies-are-hardto-scrap/2012/06/18/gJQABaQUlV_blog.html
% http://www.forbes.com/sites/timworstall/2012/09/17/must-we-again-about-fossil-fuel-subsidies-around-the-world/

There are subsidies for bio-diesel which the EPA estimates to have a negative
effect on the environment.\FIXME{Follow up on: Sophie E. Miller, When
Environmental Quality Isn’t the Goal: New EPA Fuel Standards Foul the Air,
Regulatory Studies Center: George Washington University, (Oct. 31, 2012)
\url{http://regulatorystudies.gwu.edu/images/commentary/epa_rfs_2013.pdf} (commenting
on Regulation of Fuels and Fuel Additives: 2013 Biomass-Based Diesel Renewable
Fuel,} Some will say that the problem was that the matter was not well-studied.
In fact, the problem was the model: trying to encourage a specific technology
with the promise that it would be more efficient rather than directly attacking
CO${}_2$ usage (which is the real problem).

However, there are still many subsidies being given to the worst possible use:
keeping coal mines open; and for the worst possible reason: to prevent job
losses amongst coal miners. Even supposedly environmentally-friendly Germany
spends about 4 billion dollars a year keeping loss-making coal mines in
business. In fact, they even spent some political capital in European Union
negotions to make sure that their government was allowed to continue doing
this. Spain, too, another green-energy champion, was, until recently spending
billions per year to keep coal mines open.\footnote{It was the hole in the
budget that finally caused the government to change its mind, not a change of
heart on the environmental damage from this policy. As I write this, the policy
change has led to massive protests (which have garnered the support of some
left-wing commentators!).}

When it comes to the environment and government policy, a good place to start
would be ``first, do no harm.''

%http://www.ft.com/intl/cms/s/0/5f1fa75e-047c-11e0-a99c-00144feabdc0.html

\section{Pollution Taxes}

A tax is good when it achieves two purposes: (1) raises revenue and (2) curbs
harmful behavior.

Taxes do not have strictly revenue raising effects. A tax on any activity will
always curb that activity. A tax on labor causes people to hire others
less. A tax on investment gains may curb investment.\footnote{The true effects
are much more complex. Who ends up really paying for a tax is complex.
Moreover, a tax on labor makes workers poorer, which may encourage them to
actually try harder to work.} A good tax is that which, if it fails to raise
revenue because people curbed that activity, will still be hailed a success.
These are sometimes called Pigovian taxes after the economist Arthur Pigou.
Harvard economist (and former George W.\ Bush advisor) Greg Mankiw has called
the informal group of economists who support a carbon tax (as he does), the
Pigou Club.

Pollution taxes are exactly such good taxes. If the result is that millions are
invested in ``tax avoidance'' by employing less polluting technologies and very
little revenue is obtained, this will be a major success.

Many small government advocates would agree with everything above, but not
actually trust that the government would fall through with the second half of
the bargain: reducing other taxes. In their view, this line of argument could
be a bait-and-switch on the part of big government liberals.\footnote{Many
thoughtful liberals feel the same way about free trade. They agree that there
is a potential for everyone to be better off, but do not trust that the
government is very effective at redistributing the gains to the losers of free
trade.} In this view, any tax increase is to be opposed as it is a tax
increase.

This is short-sighted and blindly partisan. It may be an appropriate opinion
for politicians, but not for the thinking man. If pollution taxes are a good
idea and better than other taxes, we should advocate for them. This at the same
time that we advocate for lowering other taxes with more pernicious effects (of
the simplification of the complex US tax system).

One typical (thoughtful) objection to carbon taxes is that we have no good
estimate of the societal cost of CO${}_2$ emissions. Therefore, we do not know
what a good value for the tax would be. The simple rejoinder is that, while it
is true that we do not a good value,\footnote{It is possible to get estimates
of many of the components of the harm caused by CO${}_2$ pollution, but how it
is ultimately a subjective call what the dollar value of losing a species due
to the altering of its habitat. Another major problem is how to value costs
borne by people in 2100. Often, we are comfortable demanding higher costs of
the richer folks (for example, during the tax system), but it should this apply
to the future inhabitants of the earth, who will almost certainly be much
richer than we are? In sum, the final estimates of the societal cost of
CO${}_2$ vary widely.} why should this argue for setting a value of \$0? I
might not know how much my car is worth, but I am not ready to give it away for
\$0!

\section{A Role for Government}

Naturally, the previous proposals were not a recipe for how the system could
work in the absence of any centralized authority (in fact, we needed to collect
taxes and enforce laws). However, there is a role for an even more active role
for government in providing the public good of scientific knowledge.

\subsection{Basic \& Applied Research is a Public Good\ldots}

Basic research is a public good. Ideas that can serve everybody, spawn
different industries, are a public good.

\subsection{\ldots But Product Development is Not}

In some ways, the cronyism story is a distraction. The problems with direct
subsidies to technologies are often even more fundamental.

The German government spent billions subsidizing solar power.

\subsection{Environmental Cleanup is Another Public Good}

\subsection{Energy Efficiency is Not An Important Goal}

I have nothing against energy efficiency. In fact, it is often a very good
thing. But it is not a goal, it is, possibly, a means to the real goal: that of
reducing emissions.

It is not how efficient you are that matters, but your total emissions.
\footnote{As Joshua Snyder quipped on my twitter stream ``It turns out it's
hard to convince people that what they really want is less CO2, not more
hybrids.''\backnote{This was actually quoted by Jeffrey Horn (username @jrhorn424) on February 26, 2013
\href{tweet link:/jrhorn424/status/306403054603874304}{https://twitter.com/jrhorn424/status/306403054603874304}.}
} Focusing too much on efficiency might deliver very efficient waste as when
efficient light bulbs are used to build large-scale Christmas
decorations.\FIXME{Find a ref for this}

In economics, this is called \emph{the Jevon's paradox}. Back in 1865, William
Jevons noted that increased in the efficiency of coal usage would lead to more
(not less) coal usage. This is unintuitive at first, but makes some sense when
you realize that a more energy efficient system is the same as lowering the
cost of coal. Therefore, you may end up using more coal than you initially did.

\footnote{Many economists who favor a larger government than I do, point out
that the state is the only institution who is able to take all the societal
costs into account. Therefore, the state should be able to make choices that
reduce overall costs in a way that would be impossible for the private sector
because there is no single player who can centralize all the gains like the
state can. However, we consistently observe that the political system is, in
fact, very sensitive to whether the costs fall on the state (and must be
reflected on the budget) or whether they fall onto private businesses and
individuals. Unfortunately, the real government prefers large hidden costs to
smaller visible ones. This is an instance of evaluating real market failures
against ideal government.}

\subsection{Environmental Thoughts}

\thought Pollution is over-regulated, but under-priced.

\thought A good tax both raises revenue and curbs harmful behavior.

\thought Efficiency is not opposed to environmental gains. Economically
efficient environmental policy allows for more environmental protection.

\thought If we want less of something, we should directly penalize it.
Roundabout ways of achieving it are (almost always) less efficient.

\thought The tragedy of the commons can be solved both by political/bureacratic
allocation and by privatization of the commons.

