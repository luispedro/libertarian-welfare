\chapter{Environment}
\label{chpt:environment}

The environment is not, strictly speaking, a welfare state topic. However,
bringing up environmental concerns is a common objection to small government.
For many, it is \emph{the} objection to smaller government.

However, this is a topic in which small government advocates should have the
upper hand. Markets are great at optimizing resource utilization if we let the
price system reflect all costs (including environmental costs).

Most thoughtful free-market intelectuals accept these permises and support a
carbon-tax or cap-n-trade system on CO$_2$ emissions like the one that was so
successful in curbing acid rain, as I detail in the next section.
Unfortunately, the pro-business lobby has often coopted the free-market
thinkers in this matter (as in others). This has led to a tolerance for the
most absurd global warming denialism and ``us vs.\ them'' thinking.

We need to stop accepting the idea that solving environmental problems requires
strong centralized command-and-control. In fact, command-and-control
environmentalism is not as effective as the free-market environmentalism.

Environmental problems are what economists call \emph{externalities} because
they impact people who are external to any one transaction. For example, if I
buy a toy whose production resulted in heavy metals being dumped in a river,
then both parties to the transaction (me and the company selling me the toy)
are better off. My daughter, who will most likely receive and play with the
toy, is also better off. However, there are external parties who lose: those
people who live near the river, anyone who would like to eat the fish without
harm and everyone who enjoys a clean river. This, of course, is a horrible
situation and needs to be rectified.

Where common reasoning goes wrong is to then assume that the alternative is
command-and-control by politically appointed bureacrats. This model, whereby
there is a constant negociation between polluters and the regulatory agencies
as to what is an acceptable amount of pollution does lead to a reduction in
pollution but does so in an overly costly manner, which ultimately means less
pollution than we would otherwise get.

For very toxic substances, the solution might be an outright ban. For example,
the neurological effects of lead seem to be damaging enough that we want to ban
leaded gasoline and paint. They are unsafe in even small amounts. In fact, lead
has even been recently linked to the rise and subsequent fall of urban crime in
the United States and elsewhere (the US may be ahead of the curve as it moved
early to ban leaded gasoline,
\href{http://www.motherjones.com/kevin-drum/2013/01/lead-and-crime-ill-be-melissa-harris-perry-show-sunday-10-am}{Kevin
Drum predicts} that Latin America will likewise benefit from a drop in violent
crime over the next few decades).

%http://www.motherjones.com/environment/2013/01/lead-crime-link-gasoline?page=1
% 

\subsection{The Acid Rain Success Story}

I am old enough to remember the fear that acid rain would kill off all lakes
and rivers. While some of this was fear mongering to sell newspapers.

\subsection{The Dirty Coal Failure}



\section{Subsidies for Dirty Energy}

According to the International Energy Association, subsidies for fossil fuel
extraction around the world are larger than the subsidies for clean energy.
Just eliminating these subsidies could be a big boost in preventing climate
change.\footnote{``Fatih Birol, chief
economist at the International Energy Agency, said such a move could
provide half of the carbon savings needed to stop dangerous levels of climate
change.'' (in the British newspaper, The Guardian).%
% Link http://www.guardian.co.uk/environment/2012/jan/19/fossil-fuel-subsidies-carbon-target
} Therefore, in a free-market without any subsidies either way (neither for
fossil nor for clean energy), we would expect to see less fossil fuel use.

Naturally, some of these subsidies might serve other worthwhile purposes. For
example, if you live in a cold climate and low-income families in your area can
apply for heating aid in the winter in the form of lowered gas prices. This is
a subsidy to pollute, yes, but it serves another important purpose: to make
sure that low-income families (including children in those households), do not
suffer too much due to the cold. This is a subsidy for fossil consumption that
many would the loath to cut.

% http://www.washingtonpost.com/blogs/wonkblog/post/why-775-billion-in-fossil-fuel-subsidies-are-hardto-scrap/2012/06/18/gJQABaQUlV_blog.html
% http://www.forbes.com/sites/timworstall/2012/09/17/must-we-again-about-fossil-fuel-subsidies-around-the-world/

However, there are still many subsidies being given to the worst possible use:
keeping coal mines open; and for the worst possible reason: to prevent job
losses amongst coal miners. Even supposedly environmentally-friendly Germany
spends about 4 billion dollars a year keeping loss-making coal mines in
business. In fact, they even spent some political capital in European Union
negotiant's to make sure that their government was allowed to continue doing
this. Spain, too, another green-energy champion, was, until recently spending
billions per year to keep coal mines open.\footnote{It was the hole in the
budget that finally caused the government to change its mind, not a change of
heart on the environmental damage from this policy. As I write this, the policy
change has led to massive protests (which have garnered the support of some
left-wing commentators).}

When it comes to the environment and government policy, a good place to start
would be ``first, do no harm.''

%http://www.ft.com/intl/cms/s/0/5f1fa75e-047c-11e0-a99c-00144feabdc0.html


\section{Pollution Taxes}

A tax is good when it achieves two purposes: (1) raises revenue and (2) curbs
harmful behavior.

Taxes do not have strictly revenue raising effects. A tax on any activity will
always curb that activity. A tax on labor causes people to hire others
less. A tax on investment gains may curb investment.\footnote{The true effects
are much more complex. Who ends up really paying for a tax is complex.
Moreover, a tax on labor makes workers poorer, which may encourage them to
actually try harder to work.} A good tax is that which, if it fails to raise
revenue because people curbed that activity, will still be hailed a success.

Pollution taxes are exactly such good taxes. If the result is that millions are
invested in ``tax avoidance'' by employing less polluting technologies and very
little revenue is obtained, this will be a major success.

Many small government advocates would agree with everything above, but not
actually trust that the government would fall through with the second half of
the bargain: reducing other taxes. In their view, this line of argument could
be a bait-and-switch on the part of big government liberals.\footnote{Many
thoughtful liberals feel the same way about free trade. They agree that there
is a potential for everyone to be better off, but do not trust that the
government is very effective at redistributing the gains to the losers of free
trade.}

\section{A Role for Government}

Naturally, the previous proposals were not a recipe for how the system could
work in the absence of any centralized authority (in fact, we needed to collect
taxes and enforce laws). However, there is a role for an even more active role
for government in providing the public good of scientific knowledge.

\subsection{Basic \& Applied Research is a Public Good\ldots}

\subsection{\ldots But Product Development is Not}

In some ways, the cronyism story is a distraction. The problems with direct
subsidies to technologies are often even more fundamental.

\subsection{Environmental Thoughts}

``Pollution is over-regulated, but under-priced.''

``A good tax both raises revenue and curbs harmful behavior.''

