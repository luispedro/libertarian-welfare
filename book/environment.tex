\chapter{Environment}
\label{chpt:environment}

The environment is not, strictly speaking, a welfare state topic. However,
bringing up environmental concerns is a common objection to small government.
For many, it is \emph{the} objection to smaller government.

However, this is a topic in which free market advocates should have the upper
hand. Markets are great at optimizing resource utilization if we let the price
system reflect all costs (including environmental costs).

Most thoughtful free-market intelectuals accept these permises and support a
carbon-tax or cap-n-trade system on CO$_2$ emissions like the one that was so
successful in curbing acid rain, as I detail in the next section.
Unfortunately, the pro-business lobby has often coopted the free-market
thinkers in this matter (as in others). This has led to a tolerance for the
most absurd global warming denialism and ``us vs.\ them'' thinking.

We need to stop accepting the idea that solving environmental problems requires
strong centralized command-and-control. In fact, command-and-control
environmentalism is not as effective as the free-market environmentalism at
achieving its environmental goals.

\section{Externalities}

Environmental problems are what economists call \emph{externalities} because
they impact people who are external to any one transaction. For example, if I
buy a toy whose production resulted in heavy metals being dumped in a river,
then both parties to the transaction (me and the company selling me the toy)
are better off. My daughter, who will most likely receive and play with the
toy, is also better off. However, there are external parties who lose: those
people who live near the river, anyone who would like to eat the fish without
harm and everyone who enjoys a clean river. This, of course, is a horrible
situation and needs to be rectified.

Where common reasoning goes wrong is to then assume that the alternative is
command-and-control by politically appointed bureacrats. This model, whereby
there is a constant negociation between polluters and the regulatory agencies
as to what is an acceptable amount of pollution does lead to a reduction in
pollution but does so in an overly costly manner, which ultimately means less
pollution than we would otherwise get.

For very toxic substances, the solution might be an outright ban. For example,
the neurological effects of lead seem to be damaging enough that we want to ban
leaded gasoline and paint. They are unsafe in even small amounts. In fact, lead
has even been recently linked to the rise and subsequent fall of urban crime in
the United States and elsewhere (the US may be ahead of the curve as it moved
early to ban leaded gasoline,
\href{http://www.motherjones.com/kevin-drum/2013/01/lead-and-crime-ill-be-melissa-harris-perry-show-sunday-10-am}{Kevin
Drum predicts} that Latin America will likewise benefit from a drop in violent
crime over the next few decades).

%http://www.motherjones.com/environment/2013/01/lead-crime-link-gasoline?page=1
% 

\subsection{The Acid Rain Success Story}

I am old enough to remember the fear that acid rain would kill off all lakes
and rivers. While some of this was fear mongering to sell newspapers, there was
a true problem. At the time in the US, Ronald Reagan was President and this led
to the attempt to use markets to combat harmful SO${}_2$ emissions.

The system adopted was called \emph{cap-and-trade}, which roughly runs as
follows: (1) a quantity of SO${}_2$ is defined as the acceptable amount (this
is the \emph{cap}), (2) permits are handed out to organizations, which (3) can
either use them to pollute or sell them (this is the \emph{trade} half of the
scheme).

This way, the democratic/administrative process can decide exactly how much
polutant will be emited. This should be known and set well in advance so that
economic agents may plan. Ideally, the schedule for the next 10 to 20 years
could be publicly known in advance. This also has the advantage that it is a
single number and there is less room to hide a few bodies in the details by
giving in to special interests.

Curiously, it does not matter (environmentally) how the permits are given out.
For fairness, we would ideally auctioned them off 100\%, where everyone gets an
equal chance to get one (in the past, environmental groups have even bought
permits and then let them expire without polluting, which is a perfectly valid
use of the system). At the other extreme, permits can just be given out to
favored organizations. In fact, typically, something in between is adopted:
existing organizations are grandfathered in and receive a number of permits;
while others are auctioned off.

However, in both cases, firms have the same incentives: whenever they pollute
less, they can make an extra profit; or if their competitors figure out how to
produce the same product with less pollution, they will be able to undercut
them. Therefore, they will consider poluting a cost and try to minimize it: the
environmental effects are the same independently of who gets the permits
initially as long as they can sell them on the open market.

The grandfathering aspect simply gives a very valuable piece of paper (or
rather electronic permit) to a select few, which is obviously very unfair to
others and impedes new competitors from entering the market (and even existing
competitors from gaining market share). Ideally, permits would be always
auctioned off for these reasons, For public choice reasons, it is often
necessary to offer a certain amount of grandfathering to achieve buy-in from
the existing producers (so that they lobby for cap-and-trade instead of against
it). The fraction that is auctioned off is not fixed and can increase to 100\%
with time so that the grandfathering effect fades away.

An interesting, rarely mentioned, coda to this story is that, emboldened by
this success, the US delegation to the United Nations which discussed climate
change, lobbied strongly in favor of similar market-based solutions to climate
change, often against the opinions of most other countries. However, forceful
negociation by the US made the final result, the so called Kyoto Protocol, a
treaty which catered to its pro-market prejudices. In the end, as is
well-known, the US did ratify the treaty.

\subsection{The Dirty Coal Failure}

\subsection{Fisheries Collapse}

\subsection{Endangerous Species Conservation by Selling Them}

In 2012, King Juan Carlos of Spain, was caught hunting elephants in Botswana.
In fact, the trip would have been a secret, except that the King broke his hip.
As a result of this experience, the Spanish branch of the World Wildlife Fund
(WWF) stripped the King of the title of honorary President.

Partly, the trouble with the trip was that it was unseemly for the head of
state to travel in luxury (as the guest of a wealthy business-man) during a
time of economic crisis. There was, however, also the idea that hunting
elephants (an endangered species) was an environmental crime. This was what was
behind WWF's decision to ``fire'' the King as Honorary President.

This is wrong, by \emph{hunting elephants, Juan Carlos was helping the
species and keeping it away from extinction}! This may seem counter-intuitive,
but consider that neither the pig nor the cow are in any danger of extinction.

% Check up what type of elephant Juan Carlos was hunting
% http://www.rnw.nl/africa/bulletin/big-game-hunts-defended-after-spanish-kings-disputed-trip

\section{Subsidies for Dirty Energy}

According to the International Energy Association, subsidies for fossil fuel
extraction around the world are larger than the subsidies for clean energy.
Just eliminating these subsidies could be a big boost in preventing climate
change.\footnote{``Fatih Birol, chief
economist at the International Energy Agency, said such a move could
provide half of the carbon savings needed to stop dangerous levels of climate
change.'' (in the British newspaper, The Guardian).%
% Link http://www.guardian.co.uk/environment/2012/jan/19/fossil-fuel-subsidies-carbon-target
} Therefore, in a free-market without any subsidies either way (neither for
fossil nor for clean energy), we would expect to see less fossil fuel use.

Naturally, some of these subsidies might serve other worthwhile purposes. For
example, if you live in a cold climate and low-income families in your area can
apply for heating aid in the winter in the form of lowered gas prices. This is
a subsidy to pollute, yes, but it serves another important purpose: to make
sure that low-income families (including children in those households), do not
suffer too much due to the cold. This is a subsidy for fossil consumption that
many would the loath to cut.

% http://www.washingtonpost.com/blogs/wonkblog/post/why-775-billion-in-fossil-fuel-subsidies-are-hardto-scrap/2012/06/18/gJQABaQUlV_blog.html
% http://www.forbes.com/sites/timworstall/2012/09/17/must-we-again-about-fossil-fuel-subsidies-around-the-world/

However, there are still many subsidies being given to the worst possible use:
keeping coal mines open; and for the worst possible reason: to prevent job
losses amongst coal miners. Even supposedly environmentally-friendly Germany
spends about 4 billion dollars a year keeping loss-making coal mines in
business. In fact, they even spent some political capital in European Union
negotiant's to make sure that their government was allowed to continue doing
this. Spain, too, another green-energy champion, was, until recently spending
billions per year to keep coal mines open.\footnote{It was the hole in the
budget that finally caused the government to change its mind, not a change of
heart on the environmental damage from this policy. As I write this, the policy
change has led to massive protests (which have garnered the support of some
left-wing commentators).}

When it comes to the environment and government policy, a good place to start
would be ``first, do no harm.''

%http://www.ft.com/intl/cms/s/0/5f1fa75e-047c-11e0-a99c-00144feabdc0.html

\section{Pollution Taxes}

A tax is good when it achieves two purposes: (1) raises revenue and (2) curbs
harmful behavior.

Taxes do not have strictly revenue raising effects. A tax on any activity will
always curb that activity. A tax on labor causes people to hire others
less. A tax on investment gains may curb investment.\footnote{The true effects
are much more complex. Who ends up really paying for a tax is complex.
Moreover, a tax on labor makes workers poorer, which may encourage them to
actually try harder to work.} A good tax is that which, if it fails to raise
revenue because people curbed that activity, will still be hailed a success.
These are sometimes called Pigovian taxes after the economist Arthur Pigou.
Harvard economist (and former George W.\ Bush advisor) Greg Mankiw has called
the informal group of economists who support a carbon tax (as he does), the
Pigou Club.

Pollution taxes are exactly such good taxes. If the result is that millions are
invested in ``tax avoidance'' by employing less polluting technologies and very
little revenue is obtained, this will be a major success.

Many small government advocates would agree with everything above, but not
actually trust that the government would fall through with the second half of
the bargain: reducing other taxes. In their view, this line of argument could
be a bait-and-switch on the part of big government liberals.\footnote{Many
thoughtful liberals feel the same way about free trade. They agree that there
is a potential for everyone to be better off, but do not trust that the
government is very effective at redistributing the gains to the losers of free
trade.} In this view, any tax increase is to be opposed as it is a tax
increase. This is short-sighted and blindly partisan. It may be an appropriate
opinion for politicians, but not for the thinking man. If pollution taxes are a
good idea and better than other taxes (e.g., the complex morass that are US
corporate taxes), we should advocate for them.

\section{A Role for Government}

Naturally, the previous proposals were not a recipe for how the system could
work in the absence of any centralized authority (in fact, we needed to collect
taxes and enforce laws). However, there is a role for an even more active role
for government in providing the public good of scientific knowledge.

\subsection{Basic \& Applied Research is a Public Good\ldots}

\subsection{\ldots But Product Development is Not}

In some ways, the cronyism story is a distraction. The problems with direct
subsidies to technologies are often even more fundamental.

The German government spent billions subsidizing solar power.

\subsection{Environmental Cleanup is Another Public Good}

\subsection{Environmental Thoughts}

\thought Pollution is over-regulated, but under-priced.

\thought A good tax both raises revenue and curbs harmful behavior.

\thought Efficiency is not opposed to environmental gains. Economical efficient
environmental policy allows for more environmental protection.

\thought If we want less of something, we should directly penalize it.
Roundabout ways of achieving it are (almost always) less efficient.

