\chapter{The Curious Asymmetry}

The US government is currently about 40\% of GDP.\footnote{In international
comparisons, too often one sees the US Federal Government compared with total
government spending in other countries. This makes the US seem like a
small-government country. However, in the US Federal system, state and local
governments spend a large chunk of what government spending. It is also
noteworthy that many other countries manage their spending at an even more
local level, but that is rarely mentioned in international comparisons.} 

At one end of the spectrum, we can place anarcho-capitalism at one end and
full-blown communism with force labor camps at the other. At the first end,
individuals have all rights and there are no societal rights; at the other end,
individuals have no rights and society owns them. Economically speaking, we are
close to the middle of this spectrum.\footnote{On non-economy dimensions, the
20th century has seen a steady walk from the society to the individual end of
the spectrum.}

Consider the case of another basic right, that of free-speech. I could make the
following argument:

\begin{quote}
The idea that there is a free-speech right is an illusion. It might make us
feel good, but, upon further examination, we quickly detect the falsehood.
Historically, there has never been completely free speech. We cannot shout
``fire'' in a crowded movie theater and certain forms of obscenity have always
been banned.

Or consider the criminal master-mind who, upon getting arrested, claims that
``yes, he did say that so-and-so should be killed and even offered money for
it,'' but that is purely speech. Therefore, he has a First Amendment Right to
Free Speech.

This is absurd. Society has an interest in what people say. Once you accept
this, you see that there is really no difference (except in degree) between
banning libel and banning criticism of elected officials.
\end{quote}

Naturally, we understand that the above argument is bunk. Accepting certain
restrictions to unbridled speech does not mean that there is no right to free
speech. It means that the free speech can only be regulated when equally
important rights are at stake. In institutional terms, this means that free
speech regulation has an extra layer of protection in the form of judicial
supervision that more run-of-the-mill legislation does not.

Valuing free speech does not mean that there are no limits to it, that does
limits cannot change in response to cultural changes (prohibitions on erotic
material are certainly much more relaxed than before), or that there can be no
disagreements on what is and is not an acceptable limit to free speech (as the
Citizens United ruling on campaign finance made clear).

Yet, in the area of economic rights, too often, it seems that simply pointing
out that there are necessarily limits and have always been limits to economic
rights, means that the concept is meaningless. Instead, I think that we should
interpret the concept of economic rights like we interpret free speech:
violating those rights can be necessary, it can even be a good thing to do (if
it leads to a better society), but there should be a high level of scrutiny
and, unless there are strong arguments for the violation, we should preserve
people's rights.

