\chapter{Education}%
\label{chpt:education}

In the libertarian literature, there is much discussion of what a right is and,
in philosophical terms, much is made of the difference between \emph{positive}
and \emph{negative} rights. Negative rights are those that, simply put, are
ensured if nobody bothers you. You have free speech if you can write whatever
you want without any intereference (in the United States, this right is almost
absolute). Positive rights are those rights that imply that other must help you
if you need to.

Extreme libertarians deny the second kind of rights, positive rights, as they
mean that somebody else is forced to help you. In the extreme case, it may mean
that somebody may be forced to work for your benefit. In modern societies, this
means that others may be taxed so that you receive your benefit. Therefore,
conclude hard-libertarians, you cannot if having a right to something implies
that someone else must be forced to work for you, this right is immoral.

As usual, I can intelectually understand the position of the hard libertarian
and even see some cautionary value in it. We have to be careful not to assign
too many positive rights lest we sacrifice too much liberty. However, a
cautionary tale tells us to be careful when we go out. It does not tell us
never to go out. Schooling is one of those positive rights that I accept
wholeheartedly as the beneficiaries are children.\footnote{Their parents
benefit too, of course, but they are not the main beneficiaries.}

\section{School Choice}

I agree with critics that say that ``school choice is not a panacea.'' But, who
ever said it was, or that it had to solve all problems to be a good idea? It
just has to be better than the alternatives. I could just as easily argue that
``assigned-schooling has not solved all problems,'' because it has not.

Here is a reader's comment on a recent edition of The Atlantic\footnote{The
comment is available at
\url{http://www.theatlantic.com/magazine/archive/2012/12/the-conversation/309177/}.
The original article is at
\url{http://www.theatlantic.com/magazine/archive/2012/10/a-national-report-card/309087/}.}.
Nicole Allan, the original author had argued that the improved schools in New
Orleans were due to the hits against the teachers' union and reform. J. David
Young, the reader replies:

\begin{quote}
    [T]he colorful bar graph does not provide a comparison between the public
    schools and the charter schools today, but only a comparison of the schools
    then and now. The public schools may in fact be better than the charters.
\end{quote}

This is missing the point of choice and vouchers!

In fact, many supporters of choice make the same mistake. The point is not that
voucher schools are better than public schools, or that privately run schools
are better than public schools in general. The point is that, \emph{with
choice, every school is better}. If the reader is correct and the public
schools in New Orleans are now better than the charter schools and much better
than they were before reform, then this is \textbf{an even stronger} argument
for reform.

Most of the school choice studies make this exact mistake: they compare the
kids who attend charter schools with the kids left behind at public schools.
However, the public schools are not a good control group. They are competing to
be better in a way that they would not if the charter schools did not exist.

\section{Testing}

One issue that shows up with school choice is testing. Logically, they are
completely separate issues. You can have intense testing in a purely assigned
school system and no testing in a complete school choice system. In fact, with
choice you will need less testing as we expect that parents will know whether
the school is doing a good job and at least some of them will use their
increased power to steer the school.

I think that testing has become \emph{too much of a good thing}.

\subsection{School Choice Thoughts}

\thought It's not about private vs.\ public. It's about choice and competition.

\thought The biggest effect of school choice is that it improves public schools.

\thought If charter schools are not that much better in terms of test scores
(perhaps even showing no effects), but parents and students overwhelmingly
prefer them; we should allow them.

