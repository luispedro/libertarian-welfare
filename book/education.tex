\chapter{Education}
\label{chpt:education}

\section{School Choice}

I agree with critics that say that ``school choice is not a panacea.'' But, who
ever said it was, or that it had to solve all problems to be a good idea? It
just has to be better than the alternatives. I could just as easily argue that
``assigned-schooling has not solved all problems,'' because it has not.

\section{School Choice}

Here is a reader's comment on a recent edition of The Atlantic\footnote{The
comment is available at
\url{http://www.theatlantic.com/magazine/archive/2012/12/the-conversation/309177/}.
The original article is at
\url{http://www.theatlantic.com/magazine/archive/2012/10/a-national-report-card/309087/}.}.
Nicole Allan, the original author had argued that the improved schools in New
Orleans were due to the hits against the teachers' union and reform. J. David
Young, the reader replies:

\begin{quote}
    [T]he colorful bar graph does not provide a comparison between the public
    schools and the charter schools today, but only a comparison of the schools
    then and now. The public schools may in fact be better than the charters.
\end{quote}

This is missing the point of choice and vouchers!

In fact, many supporters of choice make the same mistake. The point is not that
voucher schools are better than public schools, or that privately run schools
are better than public schools in general. The point is that, \emph{with
choice, every school is better}. If the reader is correct and the public
schools in New Orleans are now better than the charter schools and much better
than they were before reform, then this is \textbf{an even stronger} argument
for reform.

Most of the school choice studies make this exact mistake: they compare the
kids who attend charter schools with the kids left behind at public schools.
However, the public schools are not a good control group. They are competing to
be better in a way that they would not if the charter schools did not exist.

\subsection{School Choice Thoughts}

``It's not about private vs.\ public. It's about choice and competition.''

``If charter schools are not that much better in terms of test scores (perhaps
even showing no effects), but parents and students overwhelmingly prefer them;
we should allow them.''

