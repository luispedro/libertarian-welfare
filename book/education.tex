\chapter{Education}%
\label{chpt:education}

Universal education is the basis of a fair and decent society (and fairness and
decency are the basis of freedom). Unfortunately, the system has been failing.
At the top of the heap, American education is one of the best in the world; but
its benefits are not spread widely. The education system that an inner city or
poor rural child is forced through is a completely different system from the
one that helps the children of the educated and affluent.

This may sound like bleeding-heart liberalism, which will be followed by a
paean to public schooling. However, public schooling, as it is organized in
most of the United States, does not live up to its ideals. Nor is the
problem simply lack of funding (if it were, it would have been fixed long ago,
as more is spent per pupil now than before).\FIXME{reference for this spending
claim}

Nor is this a diatribe against the public school system as a whole. Most
teachers (but not all) are hard working and many schools are excellent. Most of
the excellent schools, however, are serving the upper-middle class (or the
highly-educated lower earning professionals). And In fact, framing the issue as
public vs.\ private (be it profit or non-profit private) is missing the point.
The point of the critique and the point of the right reforms.

This is about affirming a right to an education and getting there based on what
works and respects the rights of parents and children alike (both of these are
important).

\section{A Right To Education}

In the libertarian literature, there is much discussion of what a right is. In
philosophical terms, much is made of the difference between \emph{positive} and
\emph{negative} rights. Negative rights are those that, simply put, are ensured
if nobody bothers you. You have free speech if you can write whatever you want
without any interference (in the United States, this right is almost
absolute, but not completely). Positive rights are those rights that imply that
other must help you if you need to. A right to education means that someone
must educate you. In practical terms, this normally means that tax dollars are
spent on schools and teachers.

Extreme libertarians deny the second kind of rights, positive rights, as they
deny that you can force anybody else to help you. They may be compelled to help
you out of the kindness of their hearts, there may even be a moral case for it;
however, you may not force them to it (and taxation implies always a threat of
force: if you do not pay your taxes, the court system will eventually tell the
police to take your property away be force). Therefore, conclude some
hard-libertarians, if having a right to something implies that someone else
must be forced to work for you, this so-called right is immoral. If they are
feeling especially dramatic, they may even compare it to slavery.\footnote{Here
is Senator Rand Paul on the right to health-care:

\begin{quote}
With regard to the idea of whether you have a right to healthcare, you have to
realize what that implies. It's not an abstraction. I'm a physician. That means
you have a right to come to my house and conscript me.

It means you believe in slavery. It means that you're going to enslave not only
me, but the janitor at my hospital, the person who cleans my office, the
assistants who work in my office, the nurses," Paul said, adding that there is
"an implied use of force.
\end{quote}

This was cited (very critically) by Matt Welch in
\href{Reason}{http://reason.com/blog/2011/05/13/rand-pauls-slavery}, an
assumedly libertarian magazine.}

As usual, I can intelectually understand the position of the hard libertarian
and even see some cautionary value in it. We have to be careful not to assign
too many positive rights lest we sacrifice too much liberty. However, a
cautionary tale tells us to be careful when we go out in the evening. It does
not tell us never to go out.

Schooling is one of those positive rights that I accept wholeheartedly as the
beneficiaries are children.\footnote{Their parents benefit too, of course, but
they are not the main beneficiaries. In fact, we should be wary of catering too
much to the parents who just want a babysitter.} Children have a right to
education and it should be provided with taxpayer resources.

However, I do not think that this implies a need for the public school system
as it is currently organized in most of the United States.

\section{School Choice}

The most obvious alternative to the current system is a voucher-based
mechanism. In this system, parents can choose which school to send their kids
to and the schools receive funding proportional to the number of students they
attract. At the opposite end, we have the assigned-school system where children
attend a school that is assigned to them by the state. This assignment is
normally based on their home address, but sometimes other factors too.

We must realize that currently in the United States, because schooling is a
local rather than a federal issue, there is no single school system and there
are almost pure voucher systems and almost pure assigned-school systems, but
most places are somewhere in between. Within the public school system, parents
often have some choices, but not too many. It would actually be possible to
have complete school choice with only public schools.

Better off parents always have choices.\footnote{Milton Friedman wrote that
``[o]ur views in these respects are, I believe, still dominated by the small
town that had but one school for the poor and rich residents alike. Under such
circumstances, public schools may well have equalized
opportunities.''\backnote{The Role of Government in Education from
\emph{Capitalism and Freedom}, Milton Friedman, 1962.}} They choose in many
ways. Firstly, they live in neighborhoods served by good public schools. These
are, generally, the more expensive neighborhoods in a city (they are often even
outside the main city area, in the poshier suburbs). Alternatively, they can
opt out of the public system completely and send their children to a private
school, including one that caters to their educational biases and
preferences (be it religiously inspired schooling, musically-oriented, or
free-range Montessori). Some even opt out of traditional schooling completely
and home-school their kids. Home schooling used to be an enclave of the
religious fanatics who objected to the study of evolution or sexual education,
but has recently become a viable option in some places such as New York City
whose public schools are notoriously bad.\backnote{\emph{The Homeschool
Diaries} by Paul Elie, The Atlantic, October 2012,
\url{http://www.theatlantic.com/magazine/archive/2012/10/the-homeschool-diaries/309089/}.
Also, \emph{Homeschooling, City-Style} by Lisa Miller in New York, October
2012,
\url{http://nymag.com/guides/everything/urban-homeschooling-2012-10/}.}\footnote{Estimates
say that there are about the same number of children being home-schooled as
attending charter schools in the US.\backnote{\emph{Keep it in the family}, The
Economist, Dec 22 2012.\href{available
online}{http://www.economist.com/news/united-states/21568763-home-schooling-growing-ever-faster-keep-it-family}.}}
The better off parents have choices. The question is only whether we should
extend these possibilities to those parents whose incomes are not as high.

I agree with critics that say that ``school choice is not a panacea.'' But, who
ever said it was, or that it had to solve all problems to be a good idea? It
just has to be better than the alternatives. I could just as easily argue that
``assigned-schooling has not solved all problems,'' because it has not.

\subsection{What Are Vouchers?}

My father-in-law once got into an argument with a ticket inspector on a
Portuguese train. Like is the case in the US, passenger trains in Portugal are
run by the state (they are better than Amtrak, though, which is admittedly, a
low bar). Finally, as a business owner, my father-in-law asked: ``is this how
you treat your customers?'' The reply came quickly ``you are not a customer,
you are a user of the service.'' My father-in-law said he then understood there
was no hope of getting better service.

When you approach a large institution as a customer, you do not always get a
good deal. But, at the very least, you can always leave. When you approach them
as a ``user,'' you have no leverage.

You may think that this is very different from the way you are treated in your
dealings with the public school system. But, then again, you are probably at a
school you chose (through buying in the right neighborhood). Because if they
treated you wrongly, you'd leave.

In a voucher system, parents can enroll their children in any school that
accepts them (public or private, profit or non-profit) without paying for it
(or paying an amount that is income-dependent). The school receives an amount
from the government for each student it enrolls. From the point of view of the
school, it's a free-market system; from the point of view of the parents, it's
socialism.

\FIXME{mention special needs children.}

\subsection{School Choice and Public Schools}

Here is a reader's comment on a recent edition of The Atlantic\footnote{The
comment is available at
\url{http://www.theatlantic.com/magazine/archive/2012/12/the-conversation/309177/}.
The original article is at
\url{http://www.theatlantic.com/magazine/archive/2012/10/a-national-report-card/309087/}.}.
Nicole Allan, the original author had argued that the improved schools in New
Orleans were due to the hits against the teachers' union and reform. J. David
Young, the reader replies:

\begin{quote}
    [T]he colorful bar graph does not provide a comparison between the public
    schools and the charter schools today, but only a comparison of the schools
    then and now. The public schools may in fact be better than the charters.
\end{quote}

This is missing the point of choice and vouchers!

In fact, many supporters of choice make the same mistake. The point is not that
voucher schools are better than public schools, or that privately run schools
are better than public schools in general. The point is that, \emph{with
choice, every school is better}. If the reader is correct and the public
schools in New Orleans are now better than the charter schools and much better
than they were before reform, then this is \textbf{an even stronger} argument
for reform.

Most of the school choice studies make this exact mistake: they compare the
kids who attend charter schools with the kids left behind at public schools.
However, the public schools are not a good control group. They are competing to
be better in a way that they would not if the charter schools did not exist.

Or consider the following comment,taken from a long article on the public
school system in Finnland\backnote{``15 Reasons Reformers Are Looking to
Finland'',
\url{http://www.onlineuniversities.com/blog/2012/08/15-reasons-reformers-are-looking-finland/}},
a system which, by some measures, is the best in world: ``Finland offers school
choice, too; the only difference is that all the choices are the same''
This is still missing the point, \emph{because you have choice}, these schools
are not the same as the schools you'd have if there was no choice!


\subsection{More Flexible Schools}

An interesting comment by Robin Hanson, who excells at interesting, is that, in
the United States, the public sector is over-regulated in the United
States\backnote{Robin Hanson in \emph{Overcoming Bias} at
\url{http://www.overcomingbias.com/2013/06/is-govt-over-regulated.html} who
notes that ``this view suggests that being pro- or anti-regulation isn’t the
same as being pro- or anti-government.'' The context are public schools in
Finnland which combine flexibility and choice.}

\section{Testing}

% FIXME Consider removing this section of turning it into a single paragraph

One issue that shows up with school choice is testing. Logically, they are
completely separate issues. You can have intense testing in a purely assigned
school system and no testing in a school choice system. In fact, with choice
you will need less testing as we expect that parents will know whether the
school is doing a good job and at least some of them will use their increased
power to steer the school.

However, it is certainly the case that testing does show up sometimes in
connection with choice. For example, the Bill and Melinda Gates Foundation is a
proponent of both choice and measurement.\footnote{Tellingly, they are
proposing a very different sort of measurement and testing than what is
typically done: they are proposing, for example, that teacher effectiveness be
evaluated by other teachers who then coach and mentor struggling ones. Another
of their favorite ideas is that students grade their teachers. This is very
different from having students sit through hours of of multiple choice
questions.\FIXME{ find Atlantic links \& original research.}}

I think that testing has become \emph{too much of a good thing}. Some testing
does allow everybody to have an approximate idea of how well the students are
doing and I find the \emph{zero testing} radicals a bit too extreme. Testing
helps the students understand what is important and what is not and it even
helps them learn. \FIXME{find links to research on this.} However, in many
American schools, testing has gotten out of hand and wastes too much of
the student's time on testing.

I do not find the problem of \emph{teaching to the test} to be a very large
problem \emph{per se}. If you have thought a class, especially to
high-schoolers, you probably have had the dispiriting experience of finishing
an inspired explanation, asking for questions and getting a single one: ``will
this be on the test?'' If you want to teach something, you should test for it.
I also think that it is possible to test for much more than typical tests do.
Unfortunately, the testing format does encourage dumbing down the curriculum to
make it easier to test: it is easier to test a series of multiple-choice
questions rather than a long-form essay.

\subsection{School Choice Thoughts}

\thought It's not about private vs.\ public. It's about choice and competition.

\thought The biggest effect of school choice is that it improves public schools.

\thought If charter schools are not that much better in terms of test scores
(perhaps even showing no effects), but parents and students overwhelmingly
prefer them; we should allow them.

\thought Better off parents already have complete choice (through choice of
neighborhood or opting for non-public schooling), the question is whether to
extend this to all parents.

\thought School choice advocates were wrong to get mixed up in the testing
imbroglio. Choice actually weakens the case for testing.

