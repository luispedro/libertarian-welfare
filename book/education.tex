\chapter{Education}%
\label{chpt:education}

Universal education is the basis of a fair and decent society (and fairness and
decency are the basis of freedom). Unfortunately, the system has been failing.
At the top, American education is one of the best in the world; but its
benefits are not spread widely.

\section{A Right To Education}

In the libertarian literature, there is much discussion of what a right is. In
philosophical terms, much is made of the difference between \emph{positive} and
\emph{negative} rights. Negative rights are those that, simply put, are ensured
if nobody bothers you. You have free speech if you can write whatever you want
without any intereference (in the United States, this right is almost
absolute, but not completely). Positive rights are those rights that imply that
other must help you if you need to. A right to education means that someone
must educate you. In practical terms, this normally means that tax dollars are
spent on schools and teachers.

Extreme libertarians deny the second kind of rights, positive rights, as they
deny that you can force anybody else to help you. They may be compelled to help
you out of the kidness of their hearts, there may even be a moral case for it;
however, you may not force them to it (and taxation implies always a threat of
force: if you do not pay your taxes, the court system will eventually tell the
police to take your property away be force). Therefore, conclude some
hard-libertarians, if having a right to something implies that someone else
must be forced to work for you, this so-called right is immoral. If they are
feeling especially dramatic, they may even compare it to slavery.\footnote{Here
is Senator Rand Paul on the right to health-care:

\begin{quote}
With regard to the idea of whether you have a right to healthcare, you have to
realize what that implies. It's not an abstraction. I'm a physician. That means
you have a right to come to my house and conscript me.

It means you believe in slavery. It means that you're going to enslave not only
me, but the janitor at my hospital, the person who cleans my office, the
assistants who work in my office, the nurses," Paul said, adding that there is
"an implied use of force.
\end{quote}

This was cited (very critically) by Matt Welch in
\href{Reason}{http://reason.com/blog/2011/05/13/rand-pauls-slavery}.}

As usual, I can intelectually understand the position of the hard libertarian
and even see some cautionary value in it. We have to be careful not to assign
too many positive rights lest we sacrifice too much liberty. However, a
cautionary tale tells us to be careful when we go out in the evening. It does
not tell us never to go out.

Schooling is one of those positive rights that I accept wholeheartedly as the
beneficiaries are children.\footnote{Their parents benefit too, of course, but
they are not the main beneficiaries. In fact, we should be wary of catering too
much to the parents who just want a babysitter.} Children have a right to
education and it should be provided with taxpayer resources.

However, I do not think that this implies a need for the public school system
as it is currently organized in most of the United States.

\section{School Choice}

The most obvious alternative to the current system is a voucher-based
mechanism. In this system, parents can choose which school to send their kids
to and the schools receive funding proportional to the number of students they
attract. At the opposite end, we have the assigned-school system where children
attend a school that is assigned to them by the state. This assignment is
normally based on their address, but sometimes other factors too.

We must realize that currently in the United States, because schooling is a
local rather than a federal issue, there is no single school system and there
are almost voucher systems and almost pure assigned-school systems, but most
places are somewhere in between. Within the public school, parents have some
choices, but not too many.

Do note that the better off parents do have choices: they can opt out of the
public system completely and send their children to a private school, including
one that caters to their educational biases and preferences (be it religiously
inspired schooling, musically-oriented, or free-range Montessori). The question
is only whether we should extend these possibilities to those parents whose
incomes are not as high.

I agree with critics that say that ``school choice is not a panacea.'' But, who
ever said it was, or that it had to solve all problems to be a good idea? It
just has to be better than the alternatives. I could just as easily argue that
``assigned-schooling has not solved all problems,'' because it has not.

\subsection{School Choice and Public Schools}

Here is a reader's comment on a recent edition of The Atlantic\footnote{The
comment is available at
\url{http://www.theatlantic.com/magazine/archive/2012/12/the-conversation/309177/}.
The original article is at
\url{http://www.theatlantic.com/magazine/archive/2012/10/a-national-report-card/309087/}.}.
Nicole Allan, the original author had argued that the improved schools in New
Orleans were due to the hits against the teachers' union and reform. J. David
Young, the reader replies:

\begin{quote}
    [T]he colorful bar graph does not provide a comparison between the public
    schools and the charter schools today, but only a comparison of the schools
    then and now. The public schools may in fact be better than the charters.
\end{quote}

This is missing the point of choice and vouchers!

In fact, many supporters of choice make the same mistake. The point is not that
voucher schools are better than public schools, or that privately run schools
are better than public schools in general. The point is that, \emph{with
choice, every school is better}. If the reader is correct and the public
schools in New Orleans are now better than the charter schools and much better
than they were before reform, then this is \textbf{an even stronger} argument
for reform.

Most of the school choice studies make this exact mistake: they compare the
kids who attend charter schools with the kids left behind at public schools.
However, the public schools are not a good control group. They are competing to
be better in a way that they would not if the charter schools did not exist.

\section{Testing}

% FIXME Consider removing this section of turning it into a single paragraph

One issue that shows up with school choice is testing. Logically, they are
completely separate issues. You can have intense testing in a purely assigned
school system and no testing in a school choice system. In fact, with choice
you will need less testing as we expect that parents will know whether the
school is doing a good job and at least some of them will use their increased
power to steer the school.

I think that testing has become \emph{too much of a good thing}. Some testing
does allow everybody to have an approximate idea of how well the students are
doing and I find the \emph{zero testing} radicals a bit too extreme. However,
in many American schools, testing has gotten out of hand and wastes too much of
the student's time on testing.

I do not find the problem of \emph{teaching to the test} to be a very large
problem \emph{per se}. If you have thought a class, especially to
high-schoolers, you probably have had the dispiriting experience of finishing
an inspired explanation, asking for questions and getting a single one: ``will
this be on the test?'' If you want to teach something, you should test for it.
I also think that it is possible to test for much more than typical tests do.
Unfortunately, the testing format does encourage dumbing down the curriculum to
make it easier to test: it is easier to test a series of multiple-choice
questions rather than a long-form essay.

\subsection{School Choice Thoughts}

\thought It's not about private vs.\ public. It's about choice and competition.

\thought The biggest effect of school choice is that it improves public schools.

\thought If charter schools are not that much better in terms of test scores
(perhaps even showing no effects), but parents and students overwhelmingly
prefer them; we should allow them.

